\section{Conclusions}

This work is a draft of a diploma thesis. It is an evaluated outcome of
a semestral project that precedes the thesis. In the draft, a motivation
to the problem and its definition is proposed. Concretely, we deal with
a user behaviour classification problem that incorporates adversarial
nature of some of the actors. The proposed threat model introduces a set
of critical objects (here URLs) that a malicious user necessarily
employs in communication with a service. This is a fundamental building
block which imposes necessary modifications of the existing threat
models met commonly in literature.

The draft explores related work and gives a formal definition of the
problem, specifying a threat model that is inspired by the Stackelberg
Prediction Game in \cite{stackelberg_games}. However, some modifications to SPG
are proposed. Last but not least, the game definition is augmented by
each players’ actions analysis, arriving at the conclusion the ERM
framework is an excellent starting point for solving adversarial machine
learning problems and gives, when combined with game theory,
mathematical programs that are related to robust optimisation.

During thesis preparation, the draft will be enriched with the remaining
parts of players’ actions analysis. This will, \emph{hopefully}, give a
mathematical program for which an algorithm will be proposed. It goes
without saying that the algorithm will then be compared to baseline
approaches on real-world data and the experiment results will be
evaluated. For now, it seems the HTTP data from the university's DNS
logs will be used. Yet, this is a subject to change.

Finally, let us naïvely believe the proposed models and experimental
findings will serve the common good, rather then the truly malicious
actors.
