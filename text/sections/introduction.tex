\section{Introduction}

In recent years, computer science has seen an advent of powerful
algorithms that are able to learn from examples. Despite the notion of
learnable algorithms was recognised and studied in pioneering times of
the field already, its wide-range real-world applications were to be
implemented only with the presence of big available data collections and
vast memory and computational resources. Therefore, nowadays one meets
the abundance of machine learning techniques used to solve various
problems. The field spans from theoretical research to practical
applications in areas such as medical diagnosis, financial predictions
and, most importantly in case of this work, computer security.

Most of the applications coin a similar scenario: a problem is
formalised following a standard machine learning paradigm; a vast data
set is collected and a proper algorithm giving the best results is found
forming a model of the problem. However, in some applications once such
a model is deployed to a complex real-world environment, one soon
identifies the model performance deteriorates due to the key aspects of
the reality that have been omitted in the standard machine learning
point of view.

An example of such an observation is seen in computer vision. It was
found that deep neural networks that reign competitions in image
classification \cite{image_net} are prone to so called adversarial
images \cite{adversarial_examples}. In particular, the state-of-the-art image
classifiers based on deep neural networks score very well in terms of
prediction accuracy, when given genuine images. However, such a
classifier can be fooled with an image that was purposely adjusted. To
put it simply, what is seen as a unambiguous cat by a human observer can
be confidently labelled as a dog by a classifier. For instance, this
phenomenon challenges traffic sign classification used in autonomous
vehicles because it has been shown that a few well-placed stickers are
able to fool the classifier and make it mis-recognise a yield sign for a
main road sign \cite{adversarial_examples_2}.

To reflect such weakness, problems are reframed to a game-theoretic
setting in which two autonomous rational players compete while following
their mutually conflicting objectives. The aforementioned example with
images is, consequently, extended in the following way. One of the
players acts as an image classifier and aims to maximise classification
accuracy, whereas the other player, an adversary, perturbs the images to
lower prediction confidence or, better, to make the classifier
misclassify the image.

Of course, the same is seen in computer security–the field defined by
adversarial nature. Intruders desire to obfuscate a detector by
adjusting their attacks \cite{adversarial_malware}; malware is developed by
optimising an executable binary \cite{adversarial_malware_pe}, and spams are improved
statistically to avoid detection \cite{good_word_attacks}.

The task central to this work is the problem of user classification in
the adversarial setting. First, we examine the task as an instance of
the user classification problem viewed by standard machine learning
paradigm. Consequently, we show that omitting the adversarial nature in
this particular problem exposes critical weaknesses, thus we extend the
model to incorporate game-theoretic notions. As an instance of the user
classification task, we consider a detector of malicious users that is
deployed by a computer security company to see which users exploit their
public API service.
