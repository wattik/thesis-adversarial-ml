\documentclass[twoside, 11pt]{article}

%Set up
\usepackage[utf8]{inputenc}
\usepackage{graphicx}
\graphicspath{ {images/} }

%Notes
\usepackage{xcolor}
\newcommand\todo[1]{\textcolor{red}{#1}}

%margins
\usepackage[a4paper, headheight=14pt, width=145mm,top=40mm,bottom=40mm,bindingoffset=20mm]{geometry}

% fancy headers
\usepackage{fancyhdr}

\newenvironment{headerline}
    {
    \pagestyle{fancy}

	\fancyhead{}
	\fancyhead[RO,LE]{\textit{Adversarial Machine Learning in Security}}

	\fancyfoot{}
	\fancyfoot[LE,RO]{\thepage}
	}
    {
    }

% tables
\usepackage{booktabs}

%linky
\usepackage{hyperref}
\usepackage{url}

%floats
\usepackage{wrapfig}

% captions
\usepackage[font={small},labelfont={bf,sf}]{caption}

% line heigth
\renewcommand{\baselinestretch}{1.1}

%use math
\usepackage{amssymb}
\usepackage{mathtools}
\usepackage{amsmath}

%shortcuts MATHS
\let\vec\relax
\newcommand\vec[1]{\mathbf{#1}}

%\DeclareRobustCommand{\cr}[1]{ #1^{\mathsf{cr}} }
%\DeclareRobustCommand{\E}{ \mathop{\mathbb{E}} }

\newcommand\pr[1]{ #1^{\mathsf{pr}} }
\newcommand\adv[1]{ #1^{\mathsf{adv}} }

\newcommand\E{ \mathop{\mathbb{E}} }
\newcommand\argmin{\mathop{\text{argmin}}}

\newcommand\ben{\mathsf{B}}
\newcommand\mal{\mathsf{M}}

\let\H\relax
\newcommand\H{\mathcal{H}}
\newcommand\C{\mathcal{C}}
\newcommand\F{\mathcal{F}}
\newcommand\X{\mathcal{X}}
\newcommand\G{\mathcal{G}}

\newcommand\U{\mathbb{U}}
\newcommand\R{\mathbb{R}}

\newcommand\minus{ {{\text{-}1}} }
\newcommand\plus{ {{\text{+}1}} }

\newcommand\FNR{\textsf{FNR}}
\newcommand\FPR{\textsf{FPR}}

\newcommand\BR{\textbf{BR}}

% Theorems
\usepackage{amsthm}

%\newtheorem{theorem}{Theorem}[section]
%
%\newtheorem{lemma}[theorem]{Lemma}

\newtheorem{proposition}{Proposition}[section]
\newtheorem{definition}{Definition}[section]


% Optimision redefintion
\let\mid\relax
\newcommand\mid{\,|\,}

\let\left\relax
\newcommand\left{[\,}

\let\right\relax
\newcommand\right{\,]}

\newcommand\st{\qquad \text{s.t.} \quad}




% Section redefinition
% -- Needed only when using Texts to generate the pdf. Thus commented.
% \makeatletter
%
% \let\mysection\section
% \renewcommand{\section}{%
%   \@ifstar{%
%     \cleardoublepage
%     \mysection*%
%   }{%
%     \cleardoublepage
%     \mysection%
%   }%
% }
%
% \makeatother


% References
\usepackage{cite}
\usepackage[nottoc]{tocbibind}
\bibliographystyle{ieeetr} %ieeetr

% Appendices
\usepackage[page]{appendix}

% PDFfile
\usepackage[final]{pdfpages}


% Blank page
\usepackage{emptypage}


% enable full XeLaTeX power
%\usepackage{xltxtra}


% prevent overfull lines
\setlength{\emergencystretch}{3em}

% for exact placement of figures
%\usepackage{float}

% for footnotes in tables
%\usepackage{longtable}


\begin{document}
%%%%%%%%%%%%%%%%%%%%%%%%%%%%%%%%%%%%%%%%%%%%%%%%%%%%%%%%%%%%%%%%%%%%%%%%%%%%%%


\thispagestyle{empty}
\pagenumbering{gobble}
\begin{center}
	\includegraphics[width=0.15\textwidth]{ctulogo}\par\vspace{0.6cm}
	{\scshape\LARGE Czech Technical University in Prague \par}
	{\scshape Faculty of Electrical Engineering \par}
	{Department of Computer Science \par}
	\vspace{0.7cm}
	{\scshape\Large Semestral Project\par}

	\vspace{2.9 cm}
	{\Huge\bfseries Adversarial Machine Learning for Detecting Malicious Behavior in Network Security \par}
	\vspace{1.5cm}
	{\Large\itshape Bc. Michal Najman\par}

	\vspace{5cm}
	Supervised by\par
	\vspace{1mm}
	Mgr. Viliam \textsc{Lisý}, MSc., Ph.D.

	\vfill

% Bottom of the page
	{Submitted in May, 2019\par}
\end{center}


\cleardoublepage

%%%%%%%%%%%%%%%%%%%%%%%%%%%%%%%%%%%%%%%%%

\thispagestyle{plain}
\pagenumbering{roman}

%\includepdf[pages=-]{Bachelor-project-assignment.pdf}

\cleardoublepage

%%%%%%%%%%%%%%%%%%%%%%%%%%%%%%%%%%%%%%%%%%%%%%%%%%%%%%%%%%%%%%%= ABSTRACT

\begin{center}
    \textbf{\begin{Large}
    Abstract
    \end{Large}}
\end{center}
In this thesis, we elaborate on image captioning concerning especially dense image captioning. We present technical fundamentals of a model striving to solve such a task. Concretely, a detailed structure of DenseCap and Neural Image Caption is discussed. Experimentally, we examine results of DenseCap and analyse the model's weaknesses. We show that $92 \%$ of the generated captions are identical to a caption in the training set while the quality of those and the novel ones remains the same. We propose a criterion that significantly reduces a set of captions addressing an image whilst SPICE score of the set is maintained.
\\\\
\textbf{Keywords:} image captioning, dense captioning, convolutional neural networks, long short-term memory

\vspace{1cm}

\begin{center}
    \textbf{\begin{Large}
    Abstrakt
    \end{Large}}
\end{center}
Tato bakalářská práce se zaměřuje na automatickou tvorbu popisu obrázků (angl. image captioning), konkrétně na tzv. dense captioning. Problematika je ukázána ve světle současných modelů se zaměřením na stavbu DenseCap a Neural Image Caption. DenseCap zejména je prodoben experimentům, díky nimž jsou identifikovány nedostatky modelu. Pokusy ukazují, že $92 \: \%$ generovaných popisků je identických vzorkům v trénovací množině. Je zjištěno, že jejich kvalita v porovnání s těmi, které v trénovací množině nejsou, je stejná. V neposlední řadě je navrženo kritérium, pomocí něhož lze významně zmenšit množinu popisků vztahujících se ke konkrétnímu obrázku, kdy SPICE skóre této menší množiny zůstává stejné.
\\\\
\textbf{Klíčová slova:} automatická tvorba popisu obrázků, dense captioning, konvoluční neuronové sítě, long short-term memory
\\\\
\textbf{Český název:} Automatická tvorba popisu obrázku pomocí konvolučních neuronových sítí

\cleardoublepage

%%%%%%%%%%%%%%%%%%%%%%%%%%%%%%%%%%%%%%%%%%%%%%%%%% AUTHOR STATEMENT

\vspace*{\fill}

\section*{Author statement for undergraduate thesis:}

I declare that the presented work was developed independently and that I have listed all sources of information used within it in accordance with the methodical instructions for observing the ethical principles in the preparation of university theses.
\\\\
Prague, \today
\begin{flushright}
\line(1,0){150}
\\
Michal Najman $\quad \qquad$
\end{flushright}

\cleardoublepage

%%%%%%%%%%%%%%%%%%%%%%%%%%%%%%%%%%%%%%%%%%%%%%%%%% ACKNOLEDGMENT

\vspace*{\fill}
\section*{Acknowledgements}
I gratefully thank my supervisor Dr. Juho Kannala for his wise and critical comments as well as for his enriching attitude. Kiitos! Also, my appreciation goes to prof. Ing. Jiří Matas, Ph.D. who joined the thesis meetings and contributed immensely with novel ideas.

Last but not least, I thank my family for their support and my girlfriend Barbora for selecting the most pleasant shade of orange advisedly and for reviewing this text thoroughly.

\cleardoublepage

%%%%%%%%%%%%%%%%%%%%%%%%%%% TOC

\tableofcontents
\cleardoublepage

%%%%%%%%%%%%%%%%%%%%%%%%%%%%%%%%%%%%%%%%%%%%%%% ACTUAL CONTENT
\pagenumbering{arabic}
\setcounter{page}{1}

\begin{headerline}
	%%\addcontentsline{toc}{section}{Introduction}
	\section*{Introduction}

In recent years, computer science has seen an advent of powerful
algorithms that are able to learn from examples. Even though the notion of
learnable algorithms was recognised and studied in pioneering ages of
the field already, its wide-range real-world applications were to be
implemented only with the presence of big available data collections and
vast memory and computational resources. Therefore, nowadays one meets
the abundance of machine learning techniques used to solve various
problems. The field spans from theoretical research to practical
applications in areas such as medical diagnosis, financial predictions
and, most importantly in case of this work, computer security.

Most of the applications follow a similar scenario: a problem is
formalised following a standard machine learning paradigm; a vast data
set is collected and a proper algorithm giving the best results is found
forming a model of the problem. However, in some applications, once such
a model is deployed to a complex real-world environment, one soon
identifies the model performance's deteriorates due to the key aspects of
the reality that have been omitted in the standard machine learning
point of view.

An example is seen in computer vision. It was
found that deep neural networks that reign competitions in image
classification \cite{image_net} are prone to so called adversarial
images \cite{adversarial_examples}. In particular, the state-of-the-art image
classifiers based on deep neural networks score very well in terms of
prediction accuracy when given genuine images. However, such a
classifier can be fooled with an image that was purposely adjusted. To
put it simply, what is seen as an unambiguous cat by a human observer can
be confidently labelled as a dog by a classifier. For instance, this
phenomenon challenges traffic sign classification used in autonomous
vehicles because it has been shown that a few well-placed stickers are
able to fool the classifier and make it misrecognise a yield sign for a
main road sign \cite{adversarial_examples_2}.

To reflect such weakness, problems are reframed to a game-theoretic
setting in which two autonomous rational players compete while following
their mutually conflicting objectives \cite{towards_deep_learning_models, defense_gan, gan}. The aforementioned example with
images is, consequently, extended in the following way. One of the
players acts as an image classifier and aims to maximise classification
accuracy, whereas the other player, an attacker, perturbs the images to
lower prediction confidence or, even better, to make the classifier
misclassify the image.

Of course, the same is seen in computer security–-the field defined by
adversarial nature. Intruders desire to circumvent a detector by
adjusting their attacks \cite{adversarial_malware}; malware is developed by
optimising an executable binary \cite{adversarial_malware_pe}, and spams are improved statistically to avoid detection \cite{good_word_attacks}.

The aforementioned examples are instances of adversarial machine learning which is a field defined by two principal objectives: to design an attacker which is able to circumvent a classifier; and to design a classifier that is able to detect those attackers. In this work, we closely examine both aspects of adversarial machine learning and design an attacker and a detector uniquely combining machine learning and game theory.

In contrast to classical statistical learning, an adversarial setting such as network security has three critical properties: firstly, only benign activity can be recorder; secondly, malicious activity responds to the presence of a detector and is optimised to meet the attacker's goal; and thirdly, a real-world detector is allowed to falsely misclassify only a limited portion of benign users.

To address those three properties, we start with the expected risk minimisation framework \cite{vapnik} and adjust it account for a strict false positive rate constraint. We then define a model of an attacker and a detector as two competing entities that play a Stackelberg game \cite{stackelberg_games} and derive an optimisation task that builds upon statistical learning and game theory. Inspired by the state-of-the-art algorithms solving complex games \cite{stackgrad, exploitability_descent}, we propose an algorithm that gives an approximate solution to the game optimisation task, that is the algorithm outputs an adversarial detector robust to potential attacks. A critical part of our approach is an attack algorithm which is used as an opponent in the detector's algorithm and the detector learns to detect its attacks. In contrast to standard classifiers, our adversarial detector is stochastic. This means that its output is a posterior class distribution rather than a most probable class as it is done with standard classifiers. The final label is then drawn from the detector's output.

We work with a real-world example to demonstrate our algorithms: a URL reputation service is usually used by anti-malware programs deployed at an end-user's device to warn the user that it is about to enter a malicious site. However, the reputation service gets misused by malicious actors who this way check wether a newly deployed malicious site of theirs has already been exposed. Using the proposed algorithms, we solve the task and design such a robust adversarial detector that is capable of recognising whether a user using this reputation service is benign or malicious solely based on URLs it queried the reputation service with and information in the corresponding HTTP requests. This is done based on real-world data provided by Trend Micro Ltd.

To support our claims, we empirically show that the same level of robustness, which is achieved by our detector, is not reached with an anomaly detector on the provided real-world data. In particular, at the false positive rates $1\%$, $0.1\%$ and $0.01\%$, we show that the adversarial detector allows significantly lower portion of successful attacks. In addition, we show that our detector robustly detects attacks with more than $10$ low-scored URLs per day. Last but not least, we present our detector labels a few samples in the provided benign dataset as malicious with high confidence. On closer inspection, we find that those users exhibit suspicious behaviour and are likely a genuine attacker or an infected computer.

\paragraph{Structure of Thesis:}
In Background (Sec. \ref{sec:background}) and Related work (Sec. \ref{sec:related_work}), we review related work on adversarial machine learning. In Problem Analysis (Sec. \ref{sec:problem_analysis}), we identify specific requirements of adversarial machine learning and formally propose a solution to the problem of adversarial detection of malicious activity. In Game Definition (Sec. \ref{sec:game_definition}), we formally define a game in which a detector detects malicious users of a URL reputation system. Also, we propose two attack types: a good queries attack which performs straight-forward greedy attack and a gradient optimises attack's cost by composing obfuscation activity; and two detector types: an anomaly detector that omits the adversarial nature of the task and an adversarial detector that utilises it.
In Experiments (Sec. \ref{sec:experiments}), we empirically evaluate performance of proposed models and algorithms on real-world data provided by Trend Micro Ltd. In addition, we analyse the results and identify critical differences between proposed models.

	\cleardoublepage
	\section{Background}

\subsection{Risk Minimisation}

A classifier $h \in \mathcal{H}$ is a mapping
$h: \mathbb{X} \mapsto \mathbb{C}$ that determines which class
$c \in \mathbb{C}$ a sample $x \in \mathbb{X}$ belongs to. For the
purposes of this work, we only describe binary classification in the
following pages, however, the task is, naturally, expandable to a
general discrete set $\mathbb{C}$. In the classical risk theory, the
classifier $h$ is a subject to minimisation of expected risk $R(h)$
given a cost function
$\ell: \mathbb{C} \times \mathbb{C} \mapsto \mathbb{R}$.

\begin{equation}\label{eq:empirical-risk-minimisation}
R(h) = \E_{(x,c) \sim p} \left[ \ell(h(x), c) \right]
\end{equation}

Formally, the Expected Risk Minimisation (ERM) is given by:

\begin{equation}
\min_{h \in \mathcal{H}} R(h)
\end{equation}

Typically when working with binary classification, $\ell$ is consider
a \emph{1-0 loss} which assigns an equal cost of magnitude \emph{1} for
misclassifying objects. The expected risk in this case accounts only for
the rate of false positives and false negatives. If we employ \emph{1-0
loss} into the expected risk, we arrive at the following form:

\begin{equation}
R(h) = \sum_{c \in \mathbb{C}} p(c) \int_{x: h(x) \neq \mathsf{c}} p(x|c) \,  dx
\end{equation}

The integral can be considered a probability of classifying objects
$x$ to an incorrect class given a correct class $c$, ie.
$h(x) \neq c$. Let us consider binary classification in which
$\mathbb{C} = \{ \mathsf{B}, \mathsf{M} \}$ where $\mathsf{M}$
stands for a positive class (m for a malicious class) and $\mathsf{B}$
for a negative class (b for a benign class). In the context of this
work, the positive class refers to malicious activity, ie. activity that
is desired to be uncovered, and the negative class covers benign,
legitimate or normal behaviours. To conclude, the risk $R(h)$ can be
rewritten as a mixture of two types of errors: the probability of false
positives and the probability of false negatives.

\begin{equation}
R(h) = p(\mathsf{B}) \cdot p(\mathsf{M}|\mathsf{B}) + p(\mathsf{M}) \cdot p(\mathsf{B}|\mathsf{M})
\end{equation}

In practice, the probabilities are not known and, moreover, computing
the expected risk often involves intractable integrals. Therefore, the
risk is empirically estimated from observed samples. The empirical risk
$\hat{R}(h)$ estimated from a set of training samples
$T_m = \{ (x_i, c_i) \}_{i=1}^{m}$ is defined as follows:

\begin{equation}
\hat{R}_{T_m}(h) = \frac{1}{m} \sum_{(x_i, c_i) \in T_m} L(h(x_i), c_i)
\end{equation}

Vapnik \cite{vapnik} showed that with increasing $m$ the empirical risk
$\hat{R}_{T_m}(h)$ approaches $R(h)$.

\subsection{Regularisation}

When examining possible classifiers, we usually have a priori knowledge
of certain classifier instances being more suitable than others. Hence,
some classifiers $h$ correspond to models that are more likely to be
inadequate, and some are a priori preferred. The reasons may vary, but
mostly one desires to decrease models complexity to avoid over-fitting.
To capture this knowledge, a regularisation term
$\Omega_D: \mathcal{H} \mapsto \mathbb{R}$ penalising some classifiers
$h$ is often added to the risk.

\subsection{Neyman-Pearson Task}\label{sec:neyman-pearson}
The Neyman-Pearson Task is a problem in which the false negative rate (\textsf{FNR}) is minimised while the false positive rate (FPR) is maintained lower than a given threshold.

\begin{equation}\label{eq:neyman-pearson}
    \min_{f \in \F}
        \FNR(f)
        \st
        \FPR(f)
\end{equation}

	\cleardoublepage
	\section{Related Work}\label{sec:related_work}

Examining adversarial aspects of various machine learning problems has
currently been a popular topic. Mainly, this was triggered by Goodfellow et al.
\cite{adversarial_examples} who showed that neural networks are susceptible to
adversarial examples. Since then many endeavours have been carried out
to enhance neural networks or other machine learning algorithms by making them robust. Some tried to develop
a provably robust classifier \cite{provable_defenses}, while others reframed the
classification problem to incorporate aspects of game theory
\cite{stackelberg_games}. Huang et al. \cite{huang} identify that there are several assumptions in the machine learning scheme that are often violated. For instance, they consider the data distribution is non-stationary. Barreno et al. \cite{barreno} point out that some data may be generated by adversaries who play an instance of a deception game - that is they purposely adjust their actions to cover their true intention.

Despite most of the related work deals with image
classification, efforts to utilise the same notions in computer security
have been seen too \cite{adversarial_malware_pe, adversarial_malware_binary, adversarial_malware}. Susceptibility to adversarial
examples is, however, not the only weakness adversaries exploit, they
also are able to modify future training datasets in their favour
\cite{antidote}.


\subsection{Adversarial Machine Learning}

Lowd et al. \cite{good_word_attacks} explore obfuscate strategies yielding spams that
circumvent a spam filter. The authors consider attacks which are based
on adding words to a spammy e-mail, while other modifications are not
allowed. Three pools of words are defined: in the first attack
random words from a dictionary are drawn; the second attack utilises
common legitimate e-mail words;~and in the third attack, words that
are likely to appear in legitimate e-mails but are uncommon in spams are
added.

To select the final set of words with the greatest effect from one of
the three word pools, a black box threat model is used. In particular,
the attacker repeatedly calls the detector to identify words, which make
the detector label the spam as benign. As expected, the last pool of
words mentioned outperforms the others. Moreover, this shows that
additive changes to a malicious object are sufficient for obfuscating
the detector (within this domain). The authors claim they are able to
add words to spams in such a way the tested detection models do not
detect 50\% of them. We similarly design the good queries attack \ref{sec:good_queries_attack} in which we obfuscate user's activity by adding legitimate requests.

To reflect the successful attack algorithm, a defense strategy is
proposed. It is shown that a robust detector which uncovers the adjusted
spammy e-mails can be obtained by simply retraining the model on data
now containing the attacks. However, the authors comment, a repeated
obfuscation with a new set of effective words may again defeat the
detector.

A similar notion is seen in more advanced classification models. For
instance, deep neural networks are a popular class of classifiers
nowadays for their performance in a great range of fields. They were
shown to outperform other methods in image classification (ImageNet
Challenge \cite{image_net}), natural language processing \cite{transformer}
and in many other fields. However, it was found that neural networks
are susceptible to artificially crafted images. In particular, Goodfellow et al. \cite{adversarial_examples} show an adversarial example may be labeled as an
arbitrary class when accordingly adjusted. Moreover, despite the
transformation of an input image is substantially bounded, for example
by $l_\infty$ norm, classifiers based on neural networks are prone to
be circumvented anyway\cite{adversarial_examples_2}. The susceptibility to
adversarial samples follows the same observation in spam filtering – a
good classifier is not necessarily robust to test time data
manipulation.

As soon as it was recognised the neural networks contain built-in
vulnerabilities which are exploitable, endeavours to improve the
architecture were carried out. To address the weakness, some of the
following work focus on a model definition and consider possible attacks
already in the model design. This approach is summarised by Madry et al.
\cite{towards_deep_learning_models} who study adversarial examples
 in image classification.
The authors identify that expected risk minimisation (ERM) does not
necessarily give models robust to adversarially crafted samples.

Their work extends the training framework based on ERM by a threat model
in which each data point $x \in \mathbb{R}^N$ is assigned a set of
perturbations $S(x) \subseteq \mathbb{R}^N$ that is available to the
adversary. The authors work with $S_\epsilon(x)$ that contains
perturbations bounded by $l_\infty$, creating an
$\epsilon$-hyper-cube around each $x$:

\begin{equation}
S_\epsilon(x) = \{x' \in \mathbb{R}^N \, \mid \, l_\infty(x - x') \, \leq \, \epsilon \}
\end{equation}

The norm $l_\infty$ is used for simplicity and roughly represents
human-undetectable image perturbations. Other approaches, however,
consider more complex bounds that capture domain-specific constraints
\cite{adversarial_examples_glasses}.

To fully relate to an adversarial setting, Madry et al. \cite{towards_deep_learning_models} propose that
the adversary maximises the classifier's loss function $L$ by
modifying an image $x$ to an adversarial example
$x' \in S_\epsilon(x)$. This is further incorporated into the ERM
framework, arriving at a saddle point problem:

\begin{equation}
\min_{f \in \F} \, \E_{x, c} \leftb \max_{x' \in S_\epsilon (x)} \,  L(f(x'), \, c) \rightb
\end{equation}

In other words, a solution to the problem gives an optimal robust
classifier $f \in \F$ that is likely to classify all objects
$x \in \mathbb{R}^N$ and their neighbourhood $S_\epsilon (x)$
correctly. We similarly compose a saddle point problem to solve adversarial problems, however, we assume the attacker's goal is general and its utility does not only correspond to classification accuracy.

The saddle point problem given above consists of two sub-problems:
training the neural network and performing the inner maximisation.
\cite{towards_deep_learning_models} approach the training part with Stochastic Gradient
Descent (SGD) as it is commonly done in neural networks, while solving
the inner maximisation task with Projected Gradient Descent (PGD)
\cite{pgd}. They conclude the ERM framework extended by this
specific threat model gives a training method that is able to train
neural networks in the adversarial setting and to produce classifiers
robust to $l_\infty$ bounded image perturbations. In addition, they
find lower error is obtained with higher capacity models, suggesting
that a robust model requires more parameters (eg. layers in neural
networks).

To address the susceptibility to adversaries, several proposals of
neural networks enhancements were submitted at ICLR 2018. However, seven
out of nine were shown to be flawed due to following a similar
ineffective scheme of masking the gradients \cite{obfuscated_gradients}.

In their paper, Athalye et al. \cite{obfuscated_gradients} suggest there are three groups of
gradient masking: first, a non-differentiable layer is inserted between
the network layers; second, a classifier randomises its outputs; and
third, a function transforms the input in such a way backward gradient
explodes or vanishes. Showing that the submitted defensive methods
follow the schemes, the authors succeeded in circumventing 7 of 9
proposed models. Concretely, they replaced or removed defensive
non-differentiable components accordingly to estimate the gradient and
crafted adversarial samples with PGD.

In our problem, the attacker's optimisation criterion is not differentiable because the search space is discrete and hierarchically composed. Inspired by this approach, we solve the problem similarly and parametrise non-differentiable elements of the criterion with an interpolating function. We then solve the attacker's optimisation with PGD.

In addition, Athalye et al. \cite{obfuscated_gradients} suggest that randomisation of the classifier decision does not work for this only extends the iterations needed to acquire true gradient but does not increase robust aspects. We however use a stochastic detector whose decision is a realisation of a modelled posteriori probability – in other words, we too randomise the classifier's output. Importantly, we do so, because we model a mixed strategy which a detector as a player necessarily plays in an equilibrium. To support the claim, we show in Sec. \ref{sec:stochasticity_importance} that a stochastic classifier is more general than a deterministic classifier and outperforms it.

A key aspect of adversarial machine learning is a definition to what extent the attacker knows private parameters of a defender. In a white box approach, the attacker has full access to a gradient, a structure or parameters of the classifier \cite{obfuscated_gradients, provable_defenses, stackelberg_games, barreno}. In a black-box approach, the attacker gains access usually only to the output of the classifier and estimates other private setting from this output \cite{black-box, black_box_adversarial_attacks}. In our game setting, we work with a white box thread model.

\subsection{Provable robustness}

Until now, all presented efforts to improve the neural networks
susceptibility were approached empirically and usually without providing
provable defenses \cite{towards_deep_learning_models}. A method that aims to give provable
resistance to adversarial samples was proposed by Kotler et al. \cite{provable_defenses} who
examine a novel network architecture that provably classifies all
objects in a convex neighbourhood of a given image correctly. To achieve
that, Kotler et al. \cite{provable_defenses} redefine a ReLU \cite{relu} in such a way it
is not a function anymore but rather a set of linear constrains yielding
a convex polytope; i.e. a ReLU $y = \max \{0, x\}$ becomes:

\begin{align}
    y &\geq x \\
    y &\geq 0 \\
    y(l-u) &\leq -ux + ul
\end{align}

where $u$ and $l$ are an upper, respectively lower bound of $x$.
The bounds are unknown and need to be estimated for each ReLU.

With a convex relaxation of ReLU, image classification can be rewritten
as a linear program with all components of the network now being linear.
In the training process, the weights of the relaxed neural network are
optimised so that the network correctly classifies not only the input
image but also its convex embedding. More specifically, using a
$l_\infty$ norm a $\epsilon$-neighbourhood of an input sample is
embedded by a convex polytope and the network learns to disallow any
adversarial samples in it.

Solving the optimisation problem in its LP form with a standard LP
solver is not tractable due to a great number of variables needed to
express state-of-the-art deep neural networks. However, the LP can be
conveniently used to form an upper bound on robust classification
accuracy. Now, this upper bound combined with the ReLU input bounds
estimation becomes fully differentiable. The training process follows
standard SGD \cite{sgd} and gives a robust classifier that allows provably at most
6\% error on MNIST \cite{mnist}. In contrast, a classical neural architecture is
vulnerable up to 80\% error \cite{provable_defenses}.

\subsection{Optimising malware}

In contrast to image classification, the space of inputs is usually
discrete in computer security. An image can be represented as a vector
in $[0, 1]^n$, while executable binaries span a very sparse subset of
the binary space $\{0, 1\}^n$. Similarly, a set of executable source
codes in a given programming language is a sparse subset of all
character strings. Despite the theoretical difficulties several papers
address the issue. \cite{adversarial_malware} propose an obfuscate that optimises
a malicious source code by applying some of the predefined
modifications. The obfuscate method utilises the classifier’s gradient
to choose the most appropriate code modification. The set of plausible
modifications is given beforehand and allows only additive changes.
Although this significantly limits the attacker’s action space, the
authors claim reaching misclassification rates of up to 69\%.
\cite{adversarial_malware_binary} focus on static portable executables which they encode
into a binary feature indicator vectors. Again, additive modifications
are allowed only and malware is optimised with a bit gradient ascent.
\cite{adversarial_malware_pe} take a different approach to malware optimisation and
propose an agent which is trained with reinforcement learning. The agent
is given a portable executable and its goal is to choose the most
suitable modification of a piece of malware to lower the probability of
detection.

\subsection{Game-Theoretical Approach}

As already shown, the problem of adversarial samples can be modelled as
a game of two actors. However, Brückner et al. \cite{stackelberg_games} propose a more general
game model compared to those already mentioned. In particular, the
authors define the players as a classifier and a data generator
consisting of \emph{all} actors generating data – that is the second
player aggregately covers both benign and malicious actors.

This setting is explored using a game-theoretical point of view. The
authors propose a Stackelberg prediction game in which a classifier,
acting as a leader, and a data generator, acting as a follower, optimise
their actions to meet their objectives. They argue the Stackelberg
equilibrium is the most appropriate concept for trainable models,
specifically compared to the Nash equilibrium. It is so, they claim,
mainly because once a model is finalised and deployed, it is not changed
anymore and thus the attacker can potentially learn all details of the
model and adjust its actions to it.

In other words, the actions – the choice of model parameters and the
test time data generation – are not carried out simultaneously, but
instead the classifier commits to a specific parameters vector and the
attacker utilises the information about the model and adjusts its
attacking strategy accordingly. The later is modelled by a distribution
shift at test time. The data generator transforms a probability of data
$p$ to a test time data probability $\dot{p}$ which maximises its
objective function. In addition, the authors show that linear and
kernel-based models together with suitable objective functions allow
reformulating the problem to a quadratic program which yields the
optimal model parameters.

We define our game very similarly to the Stackelberg prediction game. However, we, in contrast, consider the data generated by benign users are stationary; and malicious users only adjust their data in response to the detector. Also, we assume the classifier is complex and largely non-linear, which leads to problems that do not have analytical solutions. And finally, we consider the defender necessarily plays a mixed strategy in order to follow the Stackelberg's equilibrium.

Amin et al. \cite{stackgrad} propose a gradient-based algorithm to solve a normal-form game by identifying a Stackelberg equilibrium. They assume that one player is a defender (playing a leader) and the second player is an attacker (a follower). They model a defender's mixed strategy with a parametrised distribution $D_\theta$ and update its parameters with a gradient descent. To estimate the expectation of the gradient, the authors use a Monte-Carlo method and sample random variables from corresponding distributions – for instance the action played by the defender is sampled from $D_\theta$. With convenient parametrisation of $D_\theta$, this approach provides an algorithm that solves a game in which the action spaces of both players are finite and discrete.

In this work, we follow this approach and similarly estimate the gradient of $D_\theta$ with a Monte-Carlo method. However, in contrast we deal with a game in which the defender's (or in our setting detector's) action space is infinite – it is in fact a family of possible classifiers and the mixed strategy is a probability distribution over all classifiers.

Lockhart et al. \cite{exploitability_descent} propose a general-scheme algorithm that approximately solves an extensive-form game of two players where each is defined by its policy $\pi_i$. The algorithm consists of iterating over a two-step process of both players computing best responses to the policies from the previous step and updating the policies based on the now-generated best responses. They prove that if both players employ this algorithm and update their policies to minimise exploitability, the policies converge to a Nash equilibrium.

In spite of the fact that we aim to find a Stackelberg equilibrium, we take a similar approach and construct an algorithm that in each iteration consists of generating the attacker's best response and updating the defender's mixed strategy $D_\theta$ by minimising its expected risk.

\subsection{Dataset poisoning}

The Stackelberg Prediction Game assumes the model is fixed after
deployment. In practice, however, engineers retrain the model on newly
obtained data that might better represent their population. As this
might be done periodically, the adversary shall take advantage of it and
adjust its obfuscate strategy. Concretely, Rubinstein et al. \cite{antidote}
elaborate on
poisoning anomaly detectors.

The poisoning obfuscate consists of purposely providing pre-crafted
samples to the detector over a long period of time in belief, that the
samples will create a blind spot in which all samples are considered
benign by the detector. The authors assume that the input space is
usually governed by a distribution of benign samples concentrated only
in certain areas, leaving the rest for anomalous activity. Given a
substantial amount of time, the adversary is gradually able to poison
the detector by targeting the large empty parts of the input space and
populating them with benign samples. In future retraining, the anomaly
detector may mistakenly consider those re-populated areas a new
phenomenon and label them benign. The attacker then simply crafts an
obfuscate near to the poisoned areas of the input space.

The authors present that such an obfuscate is possible with an anomaly
detector based on principal component analysis (PCA) which determines
directions of the sample space with greatest variance. Replacing
variance in PCA with median absolute deviation, which in contrary is a
robust scale estimator, their model is robust to data set poisoning and
successfully performs anomaly detection in backbone networks.

Despite dataset poisoning is an interesting problem, it is here mentioned to demonstrate other problems in adversarial machine learning that are not solved in this work.

	\cleardoublepage
	\section{Problem Analysis}

\todo{What is a proper name for this section?}

In the present state of Internet, it is common for a site owner to run
models classifying users or their behaviour. The task spans from user’s
interests specification to detecting deviating activity. Since such
applications are becoming more popular, one may expect the users to
modify their behaviours once they know they are being tracked and
classified. Moreover, behaviour modification may very well be of
rational nature, especially when a malicious user exploits loopholes or
carries out lawless activity in order to pursuit its goal.

In other words, if there is a cost for being disclosed or seen as a
certain category, the users will examine their actions to optimise for
lower cost. As a result, machine learning models of any kind aiming to
capture behaviours of those users necessarily need to have the adversary
nature incorporated in their design.

The straight-forward approach of solving this task would be to collect
many examples of both kinds of user activity; that is to asses a dataset
containing well-represented both malicious and benign users. This
approach would follow the standard ERM framework and would give an
activity classifier that minimises expected risk but omits the
adversarial nature. However, one might arrive at difficulties during the
construction of a balanced dataset for there is usually very few records
of malicious activity, disproportionally less than the collection of
normal, benign users. Also, and more importantly, the malicious actors
modify their attack vectors once their method is exposed or they
discover details concerning the detector.

Taking that into account, we consider the setting as a game of a classifier
competing with a body of malicious users. This approach necessarily
modifies the ERM framework and enhances it with game-theoretic notions.

This section first discusses the use-case which motivates this work.
Then, a suitable threat model is prosed and the game is formally
defined, supplemented with reasoning for given choices.

\todo{Emphasise the general adversarial machine learning problem as the core of this section.}
f
\todo{revisit the section introduction to inform about all subsections and refer to them}

A malicious activity detection system is essentially a classifier that classifies users based on their behaviour. This in principle is a machine learning problem of finding a classifier $f \in \F$ minimising expectation of detection loss $\ell_{-1}$. The detector is a mapping $f: \X \mapsto \C$ which takes vectors $x \ in \X$ on its input and produces a decision $d \in \C$. However, the ground variable representing a discrete input object is a user's activity history $h \in \H$ which is translated to a corresponding feature vector with a feature map $\Phi: \H \mapsto \X$.

All variables and functions related strictly to a detector are subscripted with $-1$, whereas we use $+1$ in the attacker's case. This choice follows Brückner et al \cite{stackelberg_games}.

As already mentioned, minimising the expected risk – as it is done in general classification problems – does not help to solve the problem of detection. The expected risk minimisation framework (ERM) consists of identifying an optimal classifier $f$ that minimises expectation of $\ell_\minus : \C \times \C \mapsto \R$ over the set $\X \times \C$.

 The expected risk can be formulated as a convex combination of risks conditioned on a class.Assuming there is two classes, i.e. $\C = \{ \ben, \mal \}$, the expected risk can be rewritten as a combination of the risk attained on the malicious class and the risk attained on the benign class.

\begin{definition}\label{def:risk}
    Let the risk attained on the malicious a class $\mal$, $R_\minus(f \mid \mal)$, be the expectation of the loss conditioned on class $\mal$. Let the risk attained on the benign class $\ben$, $R_\minus(f \mid \ben)$, be the expectation of the loss conditioned on a class $\ben$.

    \begin{align}
        R_\minus(f \mid \mal)  &= \E_{x} \left \ell_\minus(f(x), \mal) \mid \mal \right \\
        R_\minus(f \mid \ben)  &= \E_{x} \left \ell_\minus(f(x), \mal) \mid \ben \right
    \end{align}

\end{definition}

\begin{definition}[Detector's Expected Risk Minimisation]\label{def:erm}
    In standard classification, the optimal classifier $f^*$ is the solution of the following problem:

    \begin{equation}\label{eq:erm}
        \begin{aligned}
        & \underset{f \in \F} {\text{minimise}}
        & & p(\ben) \cdot R_\minus(f \mid \ben)
        +
        p(\mal) \cdot R_\minus(f \mid \mal)
        \end{aligned}
    \end{equation}
\end{definition}


\subsection{Specifics of Adversarial Machine Learning}

In this section, we examine adversarial machine learning in the domain of network security in general terms. The central task is to detect malicious users in the network without ideally affecting legitimate users.

The expectations in Def. \ref{def:erm} are usually estimated from a set of examples of each class. However, in the detection problem there are not enough examples of malicious behaviour and, in addition, this behaviour changes reflecting the current detector. This imposes two critical properties of adversarial machine learning:

\begin{itemize}
\item
    The priori class probabilities are not known.
\item
    An individual attacker follows its private objective and (possibly rationally) chooses actions minimising its cost.
\end{itemize}

\subsection{Property 1: Unknown Class Probabilities}

To reflect the first property, we redefine the detection problem to comply with the Neyman-Pearson Task \ref{sec:neyman-pearson}. Since we aim to detect malicious activity, the false positives comprise benign users classified as malicious. And, vice-versa, the false negatives are malicious users classified as benign.

\begin{definition}[Neyman-Pearson Task]\label{def:np_task}
    The Neyman-Pearson Task translates to minimising the expected loss conditioned on the malicious class while the expected loss conditioned on the benign class is maintained lower than a threshold.

    \begin{equation}\label{eq:erm}
        \begin{aligned}
        & \underset{f \in \F} {\text{minimise}}
        & & R_\minus(f \mid \mal) \\
        & \text{subject to}
        & & R_\minus(f \mid \ben) \leq \tau_0 \\
        \end{aligned}
    \end{equation}
\end{definition}

Using this formulation, prior class probabilities are omitted and, in addition, the task reflects the nature of security detection problems in which there is a hard constraint on false positives. In other words, we aim to find a classifier $f$ that does not affect legitimate activity but given this constraint is the best detector of malicious activity. As shown later, this form is particularly useful if the detector is a neural network.

\subsection{Property 2: Adversarial Setting}
By assuming an attacker is a rational actor that pursuits its goal, the setting of statistical learning changes to an adversarial game of two players: a detector and an attacker.

We sort sampled activity into two classes: benign ($\ben$) and malicious ($\mal$). The former is activity generated by legitimate users and the later is activity produced solely by attackers in pursuit of their objectives.

To avoid detection, each individual attacker obfuscates its primary activity by acting nearly as a legitimate user. However, if a good detector is deployed the obfuscation requires a large quantity of legitimate activity. If the obfuscation effort is too costly the attacker may seize to attack at all.

The obfuscated activity of an attacker, in consequence, is recorder by the detector and stored as an activity history based on which the detector assigns a label to it. The intuitive goal of the detector is to label legitimate activity history as benign, i.e. $\ben$, and the activity history generated by an attacker as malicious, i.e. $\mal$.

In the following sections, we closely explore and examine the motivations and aspects implied by the adversarial setting in network security.

\subsubsection{Stackelberg Game}
In practice, the detector is fixed after deployment and the choice of its particular form and parameters necessarily occurs before the deployment. This is a case of a Stackelberg game \cite{stackelberg_games} in which the detector is a leader and the attacker is a follower. For the sake of simplicity we assume the Strong Stackelberg Equilibrium is played.

In a Stackelberg game, the follower plays a best response to the leader's public strategy and the leader optimises this strategy accounting the follower's best response. In such a setting, the leader optimally plays a mixed strategy.

This means, the attacker obfuscates its activity optimally without a need of randomisation by playing a best response to detector's strategy.

As mentioned, the detector may necessarily randomise its actions to achieve the optimal cost. This translates to a detector playing a mixed strategy $\sigma(f) : \F \mapsto [ 0,1 ] $ instead of a single particular $f$.

\begin{definition}\label{def:stochastic-detector}
    A stochastic detector $D: \X \mapsto \C$ is a probability distribution $p(d \mid x )$ generating a decision $d \in \C $ conditioned on an observed sample $x \in \X$. $D_\sigma$ is given by a detector playing a mixed strategy $\sigma: \F \mapsto [0,1]$:

    \begin{equation}
        D_\sigma(d|x) = \sum_{f: f(x) = d} \sigma(f)
    \end{equation}

\end{definition}

\subsubsection{Information Available to Attacker}
We assume the attacker has full knowledge of the detector's structure and parameters.

\subsubsection{Attacker}
We model the attacker as a rational actor which plays the action minimising its expected costs. The particular form of costs and actions depends largely on the domain. Therefore, here we only present general notions defining the attacker and, later in Section \ref{sec:attacker}, we propose the attacker's model that suits the running example of attacks to a URL reputation service.

We propose all attackers follow the same objective and they differ only in their particular primary goal. That is, the activity obfuscation is practically the same task shared by all attackers and two attackers differ in what they aim to obfuscate.

For that reason, the model of an attacker considers a common body of attacker instances in which an individual attacker instance is thoroughly defined by its primary goal $g \in \G$. The common shared obfuscation function $\psi: \G \mapsto \H$ takes a primary goal $g$ on its input and maps it to an activity history $h = \psi(g)$ that obfuscates the primary goal.

\todo{obfs function maps to S(g), maybe make definition of psi}

\begin{definition}
    The attacker's risk $R_\plus : \Psi \times \F \mapsto \R$ is given as the expectation of its loss $\ell_\plus : \G \times \Psi \times \C \mapsto \R$. That is:

    \begin{equation}
        R_\plus(\psi, f) = \E_g \left \ell_\plus (g, \psi, f(\Phi(\psi(g)))) \right
    \end{equation}
\end{definition}

\todo{couple of words what the loss comprises.}

Relating to game theory, the obfuscation function $\psi$ is an attacker's action and its best response to $\sigma$ is given by minimising the attacker's expected risk $\E_{f \sim \sigma} R_\plus(\psi, f)$.

\begin{proposition}[Attacker's best response]\label{prop:br}
    The attacker's best response $\BR(\sigma)$ to a mixed strategy $\sigma$ is a set of obfuscation functions $\psi: \G \mapsto \H$ that are the minimisers of the expectation of the attacker's loss $\ell_\plus$.

    \begin{equation}
        \BR(\sigma) = \argmin_\psi \E_{g,d} \left \ell_\plus (g, \psi, d) \right
    \end{equation}
\end{proposition}

\begin{proof}
    \begin{align}
        \BR(\sigma) &= \argmin_\psi \E_{f \sim \sigma} R_\plus(\psi, f) \\
            &= \argmin_\psi \E_{f, g} \left \ell_\plus (g, \psi, f(\Phi(\psi(g)))) \right \\
            &= \argmin_\psi \sum_f \sum_g  \ell_\plus (g, \psi, f \circ \Phi \circ \psi(g))  \cdot  p(g) \cdot  \sigma(f)  \\
            &= \argmin_\psi \sum_g \sum_d \sum_{f: f \circ \Phi \circ \psi(g) = d}  \ell_\plus (g, \psi, d) \cdot  p(g)\sigma(f) \\
            &= \argmin_\psi \sum_g \sum_d \ell_\plus (g, \psi, d)  \cdot   p(g)  \cdot  \sum_{f: f \circ \Phi \circ \psi(g) = d} \sigma(f) \\
            &= \argmin_\psi \sum_g \sum_d \ell_\plus (g, \psi, d) \cdot  p(g)  \cdot D_\sigma(d \mid \Phi \circ \psi(g) ) \\
            &= \argmin_\psi \E_{g,d} \left \ell_\plus (g, \psi, d) \right
    \end{align}
\end{proof}

\subsubsection{Stochastic Detector}
The detector's pure strategy consists of a particular detector $f$. However, as proposed above, its optimal strategy is generally mixed and the detector, therefore, randomises its final decision $d \in \C$.

A mixed strategy in case of the detector is a probability distribution $\sigma : \F \mapsto [0,1]$ which assigns a probability to each particular detector $f$. The decision $d$ representing the estimated class of a sample $x$ is, consequently, a random variable whose probability distribution is the aggregate of probabilities $\sigma(f)$ for which $f(x) = d$. To capture that, we defined a decision distribution $D(d|x)$ in Def. \ref{def:stochastic-detector}.

In this work, we model $D(d|x)$ with a neural network which fruitfully allow us to bypass potentially infinite enumeration of detectors from $\F$. The detector's mixed strategy is, in conclusion, represented by the distribution $D_\theta (d|x)$ where $\theta$ is a parameters vector.

As proposed by Brückner et al. \cite{stackelberg_games}, the attacker's impact on the setting can be modelled by a distribution shift. However, in contrast to \cite{stackelberg_games}, in this work we assume only the malicious class activity is governed by adversarial objectives and benign activity is maintained unchanged irrespective to the detector's presence. Taking that into account, we define that the distribution of samples produced by attackers $p(x \mid \mal)$ is shifted in reaction to the presence of a deployed detector $D$ and changes to $\dot{p}(x \mid \mal, D)$.

In a standard classification problem, we find $f$ minimising the expected risk. In our adversarial setting, the detector is necessarily a distribution $D$ that is a solution to the Neyman-Pearson Task with a non-stationary distribution of samples $\dot{p}(x \mid \mal)$.

\begin{proposition}
    Let the attacker play a best response $\BR(\sigma)$ to a mixed strategy $\sigma$, then the detector's risk of a mixed strategy $\sigma$ attained on malicious activity, $R_\minus(\sigma \mid \mal)$, is given by the best-case expectation of its loss attained on malicious activity.

    \begin{align}
        R_\minus(\sigma \mid \mal) = \E_f \left R_\minus(f) \mid \mal \right = \min_{\psi \in \BR(\sigma) } \,
            \E_{q, d} \left \ell_\minus (d, \mal) \mid \mal \right
    \end{align}

    Similarly, the detector's risk of mixed strategy $\sigma$ attained on benign activity, $R_\minus(\sigma \mid \ben)$, is given by the expectation of its loss attained on benign activity.

    \begin{align}
        R_\minus(\sigma \mid \ben) = \E_f \left R_\minus(f) \mid \ben \right =
            \E_{h, d} \left \ell_\minus (d, \ben) \mid \ben \right
    \end{align}

\end{proposition}

\begin{proof}

For the risk of a mixed strategy attained on malicious activity, it holds that:
\begin{align}
    R_\minus(\sigma \mid \mal) &= \E_f \left R_\minus(f) \mid \mal \right \\
    &= \E_{f, x} \left \ell_\minus(f(x), \mal) \mid \mal \right \\
    &= \sum_f \sum_x \ell_\minus(f(x), \mal)  \cdot  \dot{p}(x \mid \mal)  \cdot  \sigma(f)
\end{align}

Consider a sample $x$ is generated solely by the attacker (due to the $\mal$ class in the conditional probability). We substitute $x$ for $\Phi \circ \psi (g)$. Assuming a feature map $\Phi: \H \mapsto \X$ projects each $h$ to one particular feature vector $x$ and a malicious activity history $h$ is  given by a primary goal $g$ obfuscated by a best response obfuscation function $\psi \in \BR(\sigma)$, the sum of probabilities $p(g)$ for which $\Phi \circ \psi(g) = x$ gives the non-stationary probability $\dot{p} (x \mid \mal)$.

\begin{align}
    \dot{p} (x \mid \mal) &= \sum_{h: \Phi(h) = x} \dot{p}(h \mid \mal) \\
    &= \sum_{h: \Phi(h) = x} \sum_{g: \psi(g) = h} p(g) \\
    &= \sum_{g: \Phi \circ \psi(g) = x} p(g)
\end{align}

Using the substitution and considering the best-case, we arrive at:

\begin{align}
    R_\minus(\sigma \mid \mal) &= \sum_f \sum_x \ell_\minus(f(x), \mal)  \cdot  \dot{p}(x \mid \mal)  \cdot  \sigma(f) \\
    &= \min_{\psi \in \BR(\sigma)}
        \sum_f \sum_g \ell_\minus(f \circ \Phi \circ \psi(g), \mal)  \cdot p(g)  \cdot \sigma(f) \\
    &= \min_{\psi \in \BR(\sigma)}
        \sum_g \sum_d \ell_\minus(d, \mal) \cdot p(g)  \cdot \sum_{f: f \circ \Phi \circ \psi(g) = d} \cdot \sigma(f) \\
    &= \min_{\psi \in \BR(\sigma)}
        \sum_g \sum_d \ell_\minus(d, \mal) \cdot p(g) \cdot D_\sigma(d \mid \Phi \circ \psi(g)) \\
    &= \min_{\psi \in \BR(\sigma) } \,
        \E_{q, d} \left \ell_\minus (d, \mal) \mid \mal \right
\end{align}

Similarly for the risk of a mixed strategy attained on benign activity:

\begin{align}
    R_\minus(\sigma \mid \ben) &= \E_f \left R_\minus(f) \mid \ben \right \\
    &= \E_{f, x} \left \ell_\minus(f(x), \ben) \mid \ben \right \\
    &= \sum_f \sum_x \ell_\minus(f(x), \ben)  \cdot p(x \mid \ben) \sigma(f) \\
    &= \sum_f \sum_h \ell_\minus( f \circ \Phi(h) ), \ben) \cdot  p(h \mid \ben)  \cdot \sigma(f) \\
    &= \sum_h \sum_d \ell_\minus(d, \ben)  \cdot p(h \mid \ben) \cdot  D_\sigma(d \mid \Phi(h)) \\
    &= \E_{h, d} \left \ell_\minus(d, \ben) \mid \ben \right
\end{align}
\end{proof}

\begin{definition}
    For simplicity, we interchangeably use $\sigma$ and $D_\sigma$ and $D_\theta$ as the detector's strategy. Thus:

    \begin{equation}
         R_\minus(D_\sigma \mid \cdot) = R_\minus(\sigma \mid \cdot) = R_\minus(\theta \mid \cdot)
    \end{equation}

\end{definition}


\begin{proposition}[Detector's optimisation problem]
    Let the detector minimise the expected risk attained on malicious activity, while maintaining the expected risk attained on benign activity upper-bounded by $\tau_0$. Let the attacker minimise its expected risk. Then the stochastic detector $D_\theta$ parametrised by $\theta$ and the obfuscation function $\psi$ which are the solution to the following bi-level optimisation problem are the Stackelberg equilibrium.

    \begin{equation}\label{eq:detector_optimisation}
        \begin{aligned}
        & \underset{\theta, \psi} {\text{minimize}}
        & & \E_{q, d} \left \ell_\minus (d, \mal) \mid \mal \right \\
        & \text{subject to}
        & & \E_{h, d} \left \ell_\minus(d, \ben) \mid \ben \right \le \tau_0 \\
        & & & \psi \in \argmin_{\psi'} \E_{g,d} \left \ell_\plus (g, \psi', d) \right
        \end{aligned}
    \end{equation}

\end{proposition}

\begin{proof}
 The proposition follows directly from the definitions and propositions above.
\end{proof}

\todo{This is cool as it follows directly from ERM when three assumptions are added: Neyman-Pearson, Rational Attacker, Stackelberg Game}

\todo{Smooth out the sequence of those propositions by mix-in explanations. This is a key train thought and it is important to stress it out clearly.}


\subsection{Assumption on Losses}
As it is common in ERM, we expect the detector's loss $\ell_\minus$ is a zero-one loss. This simplifies the primary objective in the detector's optimisation problem.

\begin{proposition}\label{prop:ben_loss}
    Let the detector's loss $\ell_\minus$ be a zero-one loss. Then the detector's risk attained on malicious activity $R_\minus(\theta \mid \mal)$ is the expectation of the posteriori probability of a benign class conditioned on malicious activity. Similarly for the risk attained on benign activity $R_\minus(\theta \mid \mal)$:

    \begin{align}
        R_\minus(\theta \mid \mal) &= \min_{\psi \in \BR(\sigma)} \E_g \left D_\theta(\ben \mid \Phi \circ \psi (g)) \mid \mal \right \\
        R_\minus(\theta \mid \ben) &= \E_h \left D_\theta(\mal \mid \Phi (h)) \mid \ben \right
    \end{align}
\end{proposition}

\begin{proof}
    The proof is straight-forward.

    \begin{align}
        R_\minus(\theta \mid \mal) &= \min_{\psi \in \BR(\sigma) } \,
            \E_{q, d} \left \ell_\minus (d, \mal) \mid \mal \right \\
           &= \min_{\psi \in \BR(\sigma) } \,
               \E_q \left \sum_d \ell_\minus (d, \mal) D_\theta(d \mid \Phi \circ \psi(q)) \mid \mal \right \\
           &= \min_{\psi \in \BR(\sigma) } \,
               \E_q \left D_\theta(\ben \mid \Phi \circ \psi(q)) \mid \mal \right \\
        R_\minus(\theta \mid \ben) &= \E_{h, d} \left \ell_\minus (d, \ben) \mid \ben \right \\
           &= \E_h \left \sum_d \ell_\minus (d, \ben) D_\theta(d \mid \Phi (h)) \mid \ben \right \\
           &= \E_h \left D_\theta(\mal \mid \Phi (h)) \mid \ben \right
    \end{align}

\end{proof}

The posteriori probability of the stochastic detector $D_\theta(d \mid x)$ is explicitly modelled by neural network in this work. Thus we prefer the risk explicitly contains the term. However, this does not hold generally and in some cases it is more fruitful to estimate the risk as expectation of loss values (e.g. reinforcement learning).

The same trick which was used in case of the defender cannot by applied to the attacker. The attacker's loss $\ell_\plus: \G \times \Psi \times \C \mapsto \R$ is more complex. Naturally, it consists of two components: a public and a private term. The public cost reflects the adversarial objective of escaping detection (e.g. detection probability). The private cost penalises the attacker for too costly obfuscation and is not necessarily adversarial to the detector's cost.

This also shows the game is a non-zero sum game as the private term in the attacker's loss does not have an adversarial equivalent in the detector's loss.

Following Brückner et al. \cite{stackelberg_games}, we defined the attacker as a shared body of attacker instances. However, if the attacker's loss is defined conveniently, the attacker's optimisation problem decomposes and it can be solved independently for each attacker's instance. The convenient form of the loss is shown Tab. \ref{tab:attacker_loss}.

\begin{table}
    \centering

        \begin{tabular}{|l|l|}
            \hline
            $d$    & $\ell_\plus(g, \psi, d)$         \\ \hline
            $\ben$ & $\Omega_\plus(g, \psi(g))$       \\ \hline
            $\mal$ & $L_0 + \Omega_\plus(g, \psi(g))$ \\ \hline
        \end{tabular}

    \caption{Table to test captions and labels}
    \label{tab:attacker_loss}
\end{table}

The motivation of this particular form of the loss is simple. If an attacker is detected it pays the amount $L_0$ for acquiring a new license or an account so that it is able to carry out further activity. However, the more complex activity histories it creates to obfuscate its primary goal, the more costly carrying out such activity is. This is represented by $\Omega_\plus: \G \times \H \mapsto \R$.

Recall that the obfuscation function $\psi(g)$ is constrained by the set of activity histories $S(g)$ such that $\psi(g) \in S(g)$, i.e. $\psi(g)$ can only create activity histories in $S(g)$. The set $S(g)$ in practice defines activity histories that the attacker is able to construct from $g$.

\begin{proposition}\label{prop:mal_loss}
    Let the attacker's loss be defined by Tab. \ref{tab:attacker_loss}. Let the attacker's private cost be a function $\Omega_\plus: \G \times \H \mapsto \R$. Then the attacker's best response problem of finding the optimal obfuscation function $\psi^*$ decomposes to identifying $\Psi^*(g)$ such that $\psi^*(g) \in  \Psi^*(g) \subset S(g) $ and $\Psi^*(g)$ is the set of solution to the following problem.

    \begin{equation}
        \Psi^*(g) = \argmin_{h \in S(g)} L_0 \cdot D_\sigma(\mal \mid \Phi(h)) + \Omega_\plus(g, h)
    \end{equation}

\end{proposition}

\begin{proof}
    Proposition \ref{prop:br} defines the attacker's best response problem in which the expectation is over variables $g$ and $d$.

    \begin{align}
        \psi^* \in \BR(\sigma) &= \argmin_\psi \E_{g,d} \left \ell_\plus (g, \psi, d) \right \\
        &= \argmin_\psi \E_g \E_d \left \ell_\plus (g, \psi, d) \right \label{eq:inner_exp}
    \end{align}

    Let us substitute the loss $\ell_\plus$ for its tabular form in Tab. \ref{tab:attacker_loss}. The inner expectation in Eq. \ref{eq:inner_exp} simplifies and becomes:

    \begin{align*}
        \E_d \left \ell_\plus (g, \psi, d) \right & = \Omega_\plus (g, \psi(g) ) \cdot D_\sigma(\ben \mid \Phi \circ \psi (g)) \, + \\
        & \quad + (L_0 + \Omega_\plus(g, \psi(g))) \cdot D_\sigma(\mal \mid \Phi \circ \psi (g)) \\
        & = \Omega_\plus(g, \psi(g)) \cdot ( 1 - D_\sigma(\mal \mid \Phi \circ \psi (g)) ) \, + \\
        & \quad + (L_0 + \Omega_\plus(g, \psi(g))) \cdot D_\sigma(\mal \mid \Phi \circ \psi (g)) \\
        & = L_0 \cdot D_\sigma(\mal \mid \Phi \circ \psi (g)) + \Omega_\plus(g, \psi(g))
    \end{align*}

    This gives us a simplified of the best response problem:

    \begin{equation*}
        \argmin_\psi \E_g \left L_0 \cdot D_\sigma(\mal \mid \Phi \circ \psi (g)) + \Omega_\plus(g, \psi(g)) \right
    \end{equation*}

    The best response problem now contains only $\psi(g)$ and the expectation can be decomposed. The criterion is minimised if we set $\psi(g)$ to $h$ that minimises $L_0 \cdot D_\sigma(\mal \mid \Phi (h)) + \Omega_\plus(g, h)$. However, the obfuscation function $\psi$ is constrained by $S(g)$. Taking $S(g)$ into account, we arrive at the following form.

    \begin{equation*}
        \psi^*(g) \in \Psi^*(g) = \argmin_{h \in S(g)} L_0 \cdot D_\sigma(\mal \mid \Phi (h) ) + \Omega_\plus(g, h)
    \end{equation*}

    $\Psi^*(g)$ simply denotes the solution of the optimisation problem.

\end{proof}

Having

\begin{proposition}
    Considering the losses in Propositions \ref{prop:ben_loss} and \ref{prop:mal_loss}, the detector's optimisation problems \ref{eq:detector_optimisation} becomes:

    \begin{equation*}
        \begin{aligned}
        & \underset{\theta} {\text{maximise}}
        & & \E_g \left \max_{\adv{h} \in \Psi^*(g)} D_\theta(\mal \mid \Phi (\adv{h})) \mid \mal \right \\
        & \text{subject to}
        & & \E_{h} \left D_\theta(\mal \mid \Phi (h)) \mid \ben \right \le \tau_0 \\
        & & & \Psi^*(g) = \argmin_{h' \in S(g)} L_0 \cdot D_\sigma(\mal \mid \Psi(h')) + \Omega_\plus(g, h')
        \end{aligned}
    \end{equation*}
\end{proposition}

\begin{proof}
    \todo{todo}
\end{proof}

\subsection{Anomaly Detection}
Note the aforementioned problem is related to (unsupervised) anomaly detection in which is is assumed no information about the malicious class is known. Thus, formulating the problem as anomaly detection, we aim to identify a detector for which the expected loss conditioned on benign class is lower than a threshold. The anomaly detection view of the problem is utilised for example if a detector consists of a k nearest neighbours distance estimator.

For our purposes we define anomaly detection as ...

\begin{definition}[Anomaly Detection]\label{def:anomaly_detection}
    Let the optimal anomaly detector be a distribution $D_\theta(d | x)$, solely parametrised by a vector $\theta$, if it is a solution to the following problem:

    \begin{equation}
        \begin{aligned}
        & \text{find}
        & & \theta \\
        & \text{such that}
        & & \E_h \left D_\theta(\mal \mid \Phi(h)) \mid \ben \right \leq \tau_0 \\
        \end{aligned}
    \end{equation}
\end{definition}

    \cleardoublepage
	\section{Game Definition}\label{sec:game_definition}
In the previous section, we proposed a malicious activity detection problem can be modelled as a game of two players: a detector and an adversary. The goal of the detector is to identify the best activity classifier, while the adversary seeks to optimally modify activity of malicious users in such a way they get misclassified by the classifier.

Below, we show such formulation can solve a real-world problem. We take a URL reputation service as a running example and formalise it and propose an algorithm that approximates the optimal solution.

In this work, we consider a network security company that runs a reputation
service which returns rating of a queried URL. For example, if we query the service's API with \textsf{www.google.com}, the URL is rated with high score whereas the malicious URL \textsf{www.malicious-url.com} is rated poorly. This type of a service is usually deployed by network security companies to provide their security software with access to most up-to-date database of URL ratings.

The typical usage scenario is coined as follows. A client running on an
end-user’s device encounters the user is about to enter a website. To
evaluate the danger level of the website, the client queries the API of the reputation service with the website URL. Accordingly, the client may show a warning message notifying the user of expected danger or carry out an
appropriate action.

Usually, URL rating systems aim to identify various URL danger types. Here, we focus on one particular type of malicious misuse: malware producers that asses a set of URLs which are used as communication entry-points for deployed malware
units. With one of these URLs, a unit of deployed malware is able to
receive commands and adjust its actions. However, to maintain
consistency and availability of its malware units, the malware producer
must regularly check whether any of its URLs has been exposed – by
querying the publicly available URL rating system.

We assume users access the service's API identified by a license and query it with HTTP requests. For the sake of simplicity, each request contains one URL whose reputation score is queried. The key element of an activity history, therefore, is the set of URLs the user has queried the service with. In particular, we record user's activity in a one-day time window. This means an activity history is a discrete object which contains different number of URLs for each user and captures a 24-hour activity record.

To conclude, the task is as follows: the computer security company desires to
distinguish malicious users of the URL rating service from benign ones based on the URLs each user queries the service with.

\subsection{Formal Definition}\label{sec:formal_definition}
In this section, we formally define the running example of this work which is an attack to a reputation system.

The service is queried with a URL $u \in \U$ where $\U$ is a set of all URLs. The query is a typical HTTP request with its attributes and the URL is the subject of the query. The service securely assigns each query to a user based on a license the user uses. Thus, we define an activity history $h \in \H$ as a collection of queries of the user. For example, if a user sends a sequence of queries for which we record a queried URL, an arrival timestamp, a source IP or possibly other information, this is recorded and integrally stored in a corresponding user’s activity history $h$.

\begin{equation}
(u_1, t_1, \mathsf{IP}_1, \dots), \, (u_2, t_2, \mathsf{IP}_2, \dots), \, \dots, \, (u_k, t_k, \mathsf{IP}_k, \dots) \longrightarrow h
\end{equation}

Recall that a user's activity history $h$ represents the ground object based on which the detector classifies users. Note that the inner structure of $h$ is discrete. This is problematic for attackers as there is no direct way of computing gradients with respect to $h$ or its elements.

In the previous section, we defined a malicious user posses a primary goal that thoroughly defines its individual instance. In this example, a single malicious user is defined by a private set of primary URLs $\pr{U} \subset \U$. The primary url set contains URLs which the malicious user necessarily employs to achieve its primary goal – that is to obtain the current reputation rating for each URL in $\pr{U}$. In consequence, the primary goal $g \in \G$ is defined entirely by the primary URLs.

\begin{equation}
    g = \pr{U}
\end{equation}

Given its primary URLs, a malicious user queries the service with URLs $U$ that may next to its primary URLs also contain legitimate queries which it uses to
obfuscate its activity.

\begin{equation}
\pr{U} \subseteq U
\end{equation}

Recall we assumed the attacker is a rational player thus the particular content of $U$ changes depending on the classifier. If there was no classifier and, therefore, the attacker was not motivated to adjust its behaviour, it would presumably query the service with $U$
resembling primary URLs and perhaps containing just a little
overhead, ie. $U \cong \pr{U}$. No detector also means there is no need to strategies with the values of other HTTP request properties.

Nonetheless, once there actually is a classifier deployed, implying a
cost for disclosure, the attacker rationally queries the service with additional legitimate URLs to obfuscate its primary goal. There are essentially to types of primary URLs $\pr{U}$ obfuscation: adding legitimate queries and adjusting properties of each request. Note that the attacker is required to include all $u \in \pr{U}$ which simplifies the problem. The allowed obfuscation methods limit the activity history derivable from a particular $\pr{\U}$. That is, each primary URLs set $\pr{U}$ induces a set of histories $S(\pr{U}) \subset \H$ that contains histories derivable from
$\pr{U}$ by obfuscation.

\begin{equation}\label{eq:derivable_histories}
S(\pr{U}) = \{ h \in \H \mid \pr{U} \subseteq \textsf{urls in }h \}
\end{equation}

We capture this with the obfuscation function $\psi: \G \mapsto \H $ which a malicious user employs to transform its original primary goal $g$ to an
obfuscated activity history $h$. Since a primary goal $g$ is solely defined by a primary URLs set $\pr{U}$, we can redefine the obfuscation function for this use case to: $\psi: 2^\U \mapsto \H$. The obfuscation is naturally bounded by the aforementioned types, thus:

\begin{equation}
    \psi(g) = \psi(\pr{U}) \in S( \pr{U} )
\end{equation}

\subsection{Features}\label{sec:features}
The key component of the detector is a feature map $\Phi: \H \mapsto \X$ where $\X \subset \R^N$ because the activity history space $\H$ is generally a discrete non-numerical set and the detector $D(d \mid x)$ requires numerical inputs. In our running example, $\H$ is a space of all possible HTTP request sequences that query a URL reputation system. Therefore, we need to construct a feature map $\Phi$ that ideally reassembles numerical attributes which are helpful in distinguishing malicious samples from benign ones. At the same time, however, we aim to omit spurious features that only provide false or correlated evidence. In an adversarial setting, these are for example features which the attacker's loss function does not depend on. In extreme case, the attacker can arbitrarily adjust those features so that its activity is not detected and it causes zero additional costs.

The feature map used in this work comprises a histogram and a density of URL scores, total count of queries and a request time distribution. The first is certainly a good-quality feature, the second may become a partially spurious feature and the last is absolutely arbitrary to the attacker's model we proposed.

In URL scores histogram, we sort URLs in a given activity to bins according to their scores. Features represent observed frequencies in each bin. URL scores density is a normed frequency histogram, i.e. we take frequency histogram and normalise it so that the values sum up to one. Total count simply represents the number of obfuscating requests (i.e. without requests related to a primary goal). A request time distribution is again a normed frequency histogram of query times within a day in which requests were sent.

Intuitively, this feature map points to a straight-forward attack method: add legitimate URLs until obfuscation is achieved. We call this method a good queries attack and use it as a baseline attack (more details in Sec. \ref{sec:good_queries_attack}).

The good queries attack however cannot properly distribute requests across time nor it can mix in URLs with different score values to mimic benign users activity. Therefore, we use a gradient attack that generates obfuscation activity based on a criterion gradient. This method is certainly more complex and requires interpolating the discrete history space $\H$. More details in Sec. \ref{sec:gradient_attack}.

\subsection{Attacker}\label{sec:attacker}
In context of the reputation service presented in the previous sections, an attacker instance is a malicious actor that posses a set of primary URLs $\pr{U}$ and aims to query the service to find out the reputation of each URL from $\pr{U}$.

Relating to the definition of the player attacker in Sec. \ref{sec:attacker-def}, the attacker's goal is to identify an obfuscation function
$\psi: 2^\U \mapsto \H$. However, with an assumption on a particular form of the attacker's loss, the attacker's task decomposes and instead of identifying $\psi$, the optimal goal obfuscation is a solution of the optimisation problem in Prop. \ref{prop:mal_loss}. We further defined an obfuscation algorithm $\pi$ which outputs an approximation of an optimal adversarial activity history.

That said, to reflect realistic attackers, we extend the attacker's operation space by a not-to-attack option. Such an option is needed because the solution to the problem in Prop. \ref{prop:mal_loss} gives an optimal activity history $h^*$ even if the actual cost of carrying out this activity exceeds the cost of no activity by far. Taking this notion into account, we allow the attacker to give up on its primary goal and carry out no activity. This activity is denoted by a token \NA. Accordingly, the codomain of an obfuscation function $\psi$, the attacker's loss and the decomposition of the optimisation problem adjust to this extension.

In effect, this means that given a primary URL set $\pr{U}$, the attacker solves the optimisation problem in Prop. \ref{prop:mal_loss} and checks whether the value of the solution is lower than the detection cost $L_0$. If it is lower, it carries out the optimal activity history. If the value of the optimum is larger than $L_0$ does not generate any activity (\NA).

\subsubsection{Attacker's Private Loss}\label{sec:attacker_loss}
The attacker's loss (as in Prop. \ref{prop:mal_loss}) has two components: a public term and a private term. The public term is a single value $L_0$ that is paid if the attacker is detected. The private term $\Omega_\plus(g, h)$ is undefined and relates to the specifics of the particular problem domain. Relating to the URL reputation system, we propose a private loss $\Omega_\plus(\pr{U}, U)$ which reflects only the number of queries the attacker produces to obfuscate its primary goal $g$, i.e. the attacker pays an amount $L_u$ for each extra legitimate URL it uses as a disguise.

\begin{equation}
    \Omega_\plus(\pr{U}, U) = L_u \cdot (|U| - |\pr{U}|)
\end{equation}

The particular value of $L_u$ is again domain- and case-dependent. To find a reasonable value, we use a following reasoning: an activity history $h$ that is labelled as $0\%$ malicious (i.e. $D_\theta(\mal \mid \Phi(h)) = 0$) costs exactly $L_0$ when it contains $\frac{L_0}{L_u}$ additional obfuscation URLs. Below, in Sec. \ref{sec:training_set}, we propose primary goals for this running example. Those primary goals contain from 1 to 40 URLs. We propose to limit the attacker to produce at most 2,000 additional URLs to construct an obfuscation activity that obfuscates a primary goal of at most $40$ URLs. This gives a relation between $L_u$ and $L_0$.

\begin{equation}
    L_u = \frac{L_0}{2000}
\end{equation}

\subsection{Good Queries Attack}\label{sec:good_queries_attack}
In Def. \ref{def:response_algorithm} we proposed that an approximative approach can be used to obfuscate primary goals. We take inspiration in Lowd et al. \cite{good_word_attacks} and propose a base line algorithm that does not give an optimal solution but may carry out a successful attack. This attack is based on the assumptions that legitimate URLs very well obfuscate primary URLs $\pr{U}$. This means, we keep adding legitimate URLs to the resulting activity history as long as it decreases the attacker's optimisation criterion. The final activity history consists of primary URLs $\pr{U}$ and the appropriate number of URLs from $V$. The remaining request parameters are set randomly.

\begin{algorithm}[H]
    \SetAlgoLined
    \KwIn{$D_\theta(\mal \mid x), \pr{U} \subset \U$, legitimate URLs $V \subset \U$}
    $U \gets \pr{U}$\;

    \While{ $D_\theta(\mal \mid \Phi(U))$ decreases}{
        arbitrarily select $u \in V$\;
        $U \gets U \cup \{u\}$\;
    }

    \Return \texttt{CreateActivityHistory}$(U)$

    \caption{Good Queries Attack}
\end{algorithm}

\subsection{Gradient Attack}\label{sec:gradient_attack}
In this section, we propose the gradient attack algorithm $\pi$ (in accordance to an attack algorithm in Def. \ref{def:response_algorithm}) that approximately obfuscates $\pr{U}$ in $T$ iterations by descending the criterion $L_0 \cdot D_\theta(\mal \mid \Phi(h)) + \Omega_\plus(\pr{U}, h)$ along its gradient (as given in Prop. \ref{prop:mal_loss}).

Notice the game of a detector $D_\theta$ and an attacker operates in three layers of spaces: internally the detector infers its decisions in a space $\X \subset \R^N$ but practically it does so utilising a feature map $\Phi$ with a discrete space $\H$ on its input. And thirdly, the attacker's algorithm $\pi$ obfuscates primary goals from a discrete space $\G$. The spaces $\X$, $\H$ and $\G$ (or in our running example $\R^N$, requests space and URL space $\U$ respectively) are entirely distinct.

In the gradient attack, we want to take gradient of $D_\theta(\mal | x )$ with respect to inputs and use it to optimally obfuscate the primary URLs $\pr{U}$. To do so, we introduce a space $K \subset \R^L$ that is an attack parameter space and a mapping $\varphi  :2^\U \times K \mapsto \R^N$ that, using primary URLs $\pr{U} \subset \U$ and an attack parameterisation $k \in K$, composes a feature vector $x \in \R^N$ such that the intermediate corresponding activity history $h$ meets the constraints imposed by the set of derivable histories $S(\pr{U})$. This is specifically useful in case of the attacker's optimisation criterion. Because if we substitute a feature map $\Phi$ for $\varphi$, we arrive this way at an optimisation task with a search space now being $K \subset \R^L$. Note that a particular form of $\varphi$ is dependent on $\Phi$.

\begin{proposition}\label{prop:differentiable_attacker}
    Let $V \subset \U$ be a set of URLs and let $\varphi  :2^\U \times K \mapsto \R^N$ be an attack parametrisation function that is differentiable in $k = [k^A, k^B] \in K \subset \R^L$. The attacker's separated optimisation task becomes:

    \begin{equation*}
        \begin{aligned}
        & \underset{k} {\text{minimise}}
        & & L_0 \cdot D(\mal \mid \varphi (\pr{U}, k)) + \Omega_\plus(\pr{U}, k) \\
        & \text{subject to}
        & & \sum_j k_j^B = 1 \\
        & & & k_i^A \in \N
        \end{aligned}
    \end{equation*}

\end{proposition}

The introduction of $\varphi$ is inspired by Athalye et al. \cite{obfuscated_gradients} who show that a non-differentiable layer in a neural net can be interpolated. They substitute such a layer for a differentiable one with similar properties and successfully compute the gradient.

\subsubsection{Attack Parametrisation}\label{sec:attack_parametrisation}
Given a particular feature map $\Phi$, it is critical to find $K$ and $\varphi$ that are ideally able to construct any $x \in \X$. This is understandably not always possible. With the feature map presented above, we therefore take the following to identify $K$ and $\varphi$.

We construct a rich enough set $V \subset \U$ which contains URLs. We associate each $u_i \in V$ with a variable $k^A_i \in \N$ which denotes that the URL $u_i$ shall be used $k^A_i$ times in the activity history that obfuscates the primary URLs. This creates a mixture of ULRs that adjusts the score histogram and the total count of requests in the feature map.

In terms of the request time entropy in the feature map, we assume it is computed over bins representing a time interval. Thus, we associate each bin $j$ with a variable $k_j^B \in [0, 1]$ that reflects a relative request mass in this bin. Naturally, the variables $k_j^B$ are normalised: $\sum_j k_j^B = 1$. With such attack parametrisation we are able to compute gradient of the criterion with respect to $k$ and construct activity histories in $S(\pr{U})$ if we have a rich enough set $V$.

The attack parametrisation function then arranges requests according to $k = [k^A, k^B]$ drawing URLs from $V$ and then computes a feature vector $x$ as if it was done with a feature map $\Phi$. Notice that we constructed $\varphi_V$ to account for the derivable histories set $S(\pr{U})$ (as defined in Eq. \eqref{eq:derivable_histories}).

\paragraph{Elements of $V$}
Ideally, we construct the set $V$ so that it contains URLs that are independent in terms of their influence on a feature map. As mentioned, the feature map $\Phi$ we use in this work constructs features based on a reputation scores histogram, a request count and request time entropy. The selection of $V$ influences only the reputation scores histogram. Therefore, we construct $V$ so that it contains URLs which each populates one bin of the reputation scores histogram. Such $V$ creates a rich-enough mixture using which we are able to construct any activity history in $S(\pr{U})$.

\subsubsection{Gradient Attack Algorithm}\label{sec:gradient_attack_algorithm}
Prop. \ref{prop:differentiable_attacker} gives a non-linear optimisation with a differentiable criterion. However, the search space is constrained by $\sum_j k_j^B = 1 $ and $k_i^A \in \N$. To solve this problem we use the projected gradient descent (PGD) \cite{pgd} combined with the fast gradient sign method (FGSM) \cite{adversarial_examples}. The attack algorithm $\pi$ of the gradient attack is shown below (Alg. \ref{alg:gradient_attack}).

\begin{algorithm}[H]
    \SetAlgoLined
    \KwIn{$D_\theta(\mal \mid x), \pr{U} \subset \U$, $V \subset \U$}
    $c(k) = L_0 \cdot D_\theta(\mal \mid \varphi_V(\pr{U}, k)) + \Omega_\plus(\pr{U}, k)$\;

    $k^{(0)} \gets $ \texttt{InitK()}\;
    \For{$t = 1, 2, \dots T$ }{
        $k^{A,(t)} \gets$ \texttt{Proj}$^A(
                \nabla_{k^A} c(k)
            )$\;
        $k^{B,(t)} \gets$ \texttt{Proj}$^B(
                \nabla_{k^B} c(k),
                k^{A,(t)}
            )$\;
    }

    \Return \texttt{MakeActivityHistory}$(V, k^{(T)})$

    \caption{Gradient Attack Algorithm}\label{alg:gradient_attack}
\end{algorithm}

\paragraph{Routine \texttt{InitK()}}
Initialisation of $k^{(0)}$ is critical because the gradient attack descents along a criterion's gradient and it turns out that setting $k^{A, (0)} = 0$, i.e. starting with solely $\pr{U}$ does not converge very well. Thus we initialise $k^{A, (0)}_i$ uniformly randomly from $\{0, 1, \dots, 2000 \}$ and set $k^{B, (0)} = \frac{1}{\text{number of bins}}$ so that it starts with maximal entropy in request time distribution.

\paragraph{Routine \texttt{Proj}$^A (z)$}
Input of this routine $z$ is a gradient vetoer with respect to $k^{A}$. We take a sign of the gradient as in FGSM but we do not scale the gradient anyhow. We update $k^{A}$ accordingly and then crop values below zero. This projection ensures $k_i^{A} \in \N$.

\begin{equation}
    k_i^{A} \gets \max\{ 0, \sign(z_i) \}
\end{equation}

\paragraph{Routine \texttt{Proj}$^B(z, k^A)$}
We maintain the scale of the input gradient $z$ (i.e. the learning rate is set to 1.0), update $k^B$ standardly, crop negative values and normalise with $Z$ to sum up to one.

\begin{equation}
    k_i^B \gets \frac{\max\{ 0,  k_i^B + z_i \} }{Z}
\end{equation}

Notice, we know the current number of requests from $k^A$:

\begin{equation}
    |U| = |\pr{U}| + \sum_i k_i^A
\end{equation}

As we defined it, $k_i^B$ corresponds to relative frequency of requests sent in a time interval $i$. Since $k^B$ is arbitrary distribution after the update, we adjust it to reflect the number of requests $|U|$. First, a time bin $i$ gets  $\texttt{floor}(k_i^B \cdot |U|)$ requests assigned. Flooring causes some requests were not assigned to a bin, thus we distribute the remaining requests randomly across bins – $\delta_i \in \{0 ,1 \}$ denotes whether a bin $i$ gets assigned a remaining request. Finally, we use these assignments to compute the relative frequency $k_i^B$.

\begin{equation}
    k_i^B \gets \frac{\texttt{floor}(k_i^B \cdot |U|) + \delta_i}{|U|}
\end{equation}

\paragraph{Routine \texttt{MakeActivityHistory(k)}}
First, we build the multi-set $U$ by concatenating $\pr{U}$ and $V$ according to $k^A$.

\begin{equation}
    U \gets \pr{U} \cup \{
        \underbrace{v_1, v_1, \dots}_{k_1^A \textrm{ times}},
        \underbrace{v_2, v_2, \dots}_{k_2^A \textrm{ times}},
        \dots,
        \underbrace{v_{|V|}, v_{|V|}, \dots}_{k_{|V|}^A \textrm{ times}}
    \}
\end{equation}

Using the same procedure as in \texttt{Proj}$^B(z, k^A)$ we assign URLs from $U$ to time bins. Finally, we create requests that each contains a URL $u \in U$ and is sent at the time associated with the bin that $u$ belongs to. We return the activity history $h$ which comprises these requests.

\subsubsection{Imperfection of Gradient Attack}

Descending along gradient is tricky, especially when projection is involved, as the descent may end up in a local minimum. The gradient attack algorithm solves the task of finding an optimal activity history $h \in \Psi^*_g$ and thus yielding $h$ which is a local optimum is problematic. However, as we adopted a rather agent-driven view of the game (as in Prop. \ref{prop:approximative_task}) we think of the gradient attack algorithm $\pi_T$ as a feasible agent that does its best to solve the task. In spite of these imperfections, we then train the detector to play against such approximative adversaries.

In the following section we propose an algorithm that solves the detector's optimisation problem.

\subsection{Detector}
In this section, we introduce two types of a detector. The first type is an anomaly detector based on $k$ nearest neighbours which solves the task in Def. \ref{def:anomaly_detection}. The second type is a stochastic detector modelled with a neural network that solves the task in Prop. \ref{prop:approximative_task}.

The output of both detectors is purposely stochastic. That is, they model the posteriori distribution $p(d \mid x)$ where $d \in \C$ is a decision and $x \in \X$ is a feature vector with a model $D_\theta(d \mid x)$. At test time, a realisation of a final label $d$ is drawn from $D_\theta(d \mid x)$. At train time, the values of probability $D_\theta(d \mid x)$ are used in the training process.

\subsubsection{Anomaly Detector}\label{sec:knn_detector}
The task of detecting malicious behaviour can also be formulated as an anomaly detection problem (as in Def. \ref{def:anomaly_detection}). We collect examples of benign behaviour and then construct an anomaly detector whose false positive rate equals $\tau_0$. This approach omits entirely the attacker's model and is based on an anomaly measure $d_k(x)$. We assume more anomalous, i.e. malicious, samples are prone to higher values of $d_k(x)$.

There are various types of anomaly detectors from which we pick one: $k$ nearest neighbours ($k$-NNs). We use average euclidian distance to $k$ nearest samples $P_k(x) \subset T^\ben$ in the training set $T^\ben$ as an anomaly measure $d_k(x)$.

\begin{equation}
    d_k(x) = \frac{1}{k} \sum_{x' \in P_k(x)} || x - x' ||_2
\end{equation}

To comply with a stochastic detector definition, we use the anomaly measure $d_k(x)$ to derive the posteriori probability $D_\alpha(\ben \mid x)$ as follows:

\begin{equation}\label{eq:knn_detector}
    D_\alpha(\ben \mid x) = \exp(-\frac{d_k(x)^2}{\alpha^2})
\end{equation}

The parameter $\alpha \in \R$ adjusts sensitivity to $x$ and is equivalent in terms of the false positive rate to a threshold on the anomaly measure that is usually used with $k$-NNs. Thus, redefining $k$-NNs to be a stochastic anomaly detector is redundant in practice, however, we do it anyway as it is convenient for comparison purposes with a true stochastic detector.

The constraint on the false positive rate in Def. \ref{def:anomaly_detection} suggests that our task is to find $\alpha$ for which the false positive rate equals $\tau_0$. We use fast gradient sign method (FGSM) to find the optimal $\alpha$ by minimising the following problem on training samples $T_m = \{\Phi(h_i)\}^m_i=1$:

\begin{equation}
    \min_\alpha \, (\frac{1}{m}\sum_{x_i \in T_m} D_\alpha(\mal \mid x_i) - \tau_0)^2
\end{equation}

The gradient attack in Alg. \ref{alg:detector_algorithm} requires the detector $D_\alpha(\ben \mid x)$ to be differentiable in $x$. $D_\alpha(\ben \mid x)$ is differentiable up to $d_k(x)$. $d_k(x)$ is not continuous in those $x$ for which $P_k(x)$ changes its elements. We estimate the gradient of $d_k(x)$ by simply taking derivative while keeping the set of $k$ nearest neighbours $P_k(x)$ fixed.

\begin{equation}
    \nabla_x d_k(x) = \frac{1}{k} \sum_{x' \in P_k(x)} \frac{x - x'}{ || x - x' ||_2 }
\end{equation}

In our experiments, we use $k=5$ as this is empirically the best value.

\subsubsection{Adversarial Detector}\label{sec:neural_detector}
We model an adversarial stochastic detector $D_\theta$ with a neural network which takes a feature vector $x \in X$ on its input and infers a probability distribution $D_\theta(d|x)$. In test time, an actual decision $d$ is drawn from the distribution $D_\theta(d|x)$. The detector's parameters $\theta$ correspond to the weights of the neural network.

We empirically arrived at a relatively shallow network consisting of five fully connected layers. Since the number of inputs $N$ is relatively low ($N~20$), we assume this is a good trade-off between network's complexity and training time. To address the non-linear nature of features we start with the first two layers being wide with $10 \cdot N$ neurons. Then we narrow the net: the third layer has $5 \cdot N$, the fourth has $5 \cdot N$. Each layer is activated with a SeLU unit. The final layer has $1$ output which is transformed with a logistic function (Eq. \eqref{eq:logistic_function}) to be bounded by $[0,1]$.

\begin{equation}\label{eq:logistic_function}
    f(z) = \frac{1}{1 + \exp(-z)}
\end{equation}

The output of the final layer's activation (i.e. logistic function) is intended to be an estimate of the posteriori probability $p(\mal \mid x)$.

\paragraph{SeLU Activations and Regularisation}
Instead of classical ReLU, we use the SeLU (Eq. \eqref{eq:selu}) activation because of better properties of its gradient and its self-normalisation effect. During the process of learning we take gradient of $D_\theta(d|x)$ with respect to $x$ to construct obfuscated activity histories. We found that near-optimal $D_\theta$ tends to adjust its weights so that initial steps of the adversarial optimisation are located in areas that are cropped but ReLU (i.e. the activations' inputs tend to be negative).

This is expected behaviour, however, as argued in \cite{obfuscated_gradients}, from the attacker's point of view this is easily bypass-able in a white-box attack. For instance, the attacker replaces all ReLUs by SeLUs. This does not change properties of the network dramatically but gives the attacker access to the gradient.

For that reason (and following the final advice of \cite{obfuscated_gradients}) we assume the attacker would gain access to gradients anyway using this trick, thus we train the net to learn to defend even such attacks and use SeLUs already. The second reason to use a SeLU as activation comes from the original paper \cite{selu} in which the SeLU was introduced. The authors prove it has weights self-normalisation properties, that is, a SeLU is able to replace batch-norm \cite{batch-norm} in a fully-connected feed forward neural nets and allows to use deep architectures with many layers.

\paragraph{Training Sets  $T^\ben$ and  $T^\mal$}\label{sec:training_set}
To identify the best parameters $\theta$ we use the detector's learning algorithm (Alg. \ref{alg:detector_algorithm}). This algorithm estimates gradients from realisations of primary goals and benign activity histories. We draw activity histories $\{ h_i \}$ from a training set $T^\ben$ we collected to capture the distribution of benign activity histories. In case of the primary goals that shall be drawn from $p(g)$, we take a different approach because the distribution $p(g) = p(\pr{U})$ is unknown. The key attributes of  the feature map $\Phi$ are based on the reputation scores of the queried URLs. We construct primary URLs sets $\pr{U}$ to reflect various ratios of already known bad-score URLs and not yet identified ones. This way we get a training set of primary urls $T^\mal$:

\begin{align*}
    T^\mal &= \{ \\
        & \quad \{ \texttt{known malicious URL} \}, \\
        & \quad \{ \texttt{uknown URL} \}, \\
        & \quad \{ \texttt{known malicious URL}, \texttt{uknown URL} \}, \\
        & \quad \{ \texttt{known malicious URL}, \texttt{known malicious URL}, \texttt{uknown URL} \}, \\
        & \quad \{ \texttt{known malicious URL}, \texttt{uknown URL}, \texttt{uknown URL} \}, \\
        & \quad \dots \\\
    &\}
\end{align*}

\paragraph{Implementation of Stochastic Detector}
We use pytorch \cite{pytorch} to implement the stochastic detector. However, due the specific requirements of the detector's learning algorithm (Alg \ref{alg:detector_algorithm}) such as the inner attack optimisation or the outer $\lambda$ double optimisation, we needed to implement the training process from scratch as the existing components of the pytorch framework does not fit the need. To compute gradients, we used the framework's autograd library, but gradient descent and the attack optimisation algorithm needed our custom implementation.

\paragraph{Handling \NA in Detector}
$T^\mal = \{ \pr{U_i}\} $ makes up a faithful mixture of reasonable primary URLs sets. In our experiments, we use a set $\pr{U_i}$ that contains at most $20$ \texttt{uknown URL}s and $20$ \texttt{known malicious URL}s. Since no prior preference over individual $\pr{U_i}$ is assumed, we draw $\pr{U_i}$ uniformly from $T^\mal$.

We allowed the attacker not to carry out any activity if obfuscation is too costly for it. This is captured by the \NA token which the attacker's algorithm $\pi$ produces instead of an activity history $h$. Despite the detector's risk is derived assuming all primary goals are translated to some activity history, the introduction of \NA does not cause principal problems as we can simply reformulate the equation for the non-stationary probability $\dot{p}(h \mid \mal)$:

\begin{equation}
    \dot{p}(h \mid \mal) = \sum_{\pr{U}: \pi(\pr{U}, D_\theta) = h} p'(\pr{U})
\end{equation}

where $p'(\pr{U})$ is the probability of observing the primary URL set for which $\pi$ does not yield \NA (i.e. $p'(\pr{U})$ is $p(\pr{U})$ normalised by the sum of $p(\pr{U})$ that are not \NA). Consequently, during the Monte-Carlo estimation of the gradient, the estimate $\gamma^\mal$ is computed from a set of obfuscated activity histories $\{ \adv{h}_i \}$. This set is generated by $\pi$ from those $\pr{U}_i \in T^\mal$ that get obfuscated, i.e $\pi(\pr{U}_i, D_\theta) \neq \NA$:

\begin{equation}
    \gamma^\mal = \frac{1}{ |\{ \adv{h}_i \}| } \sum_{j=1}^{ |\{ \adv{h}_i \}| } \nabla_\theta D_\theta(\mal \mid \Phi(\adv{h}_j))
\end{equation}

\paragraph{Similarity to Cross Entropy}
Using the Jensen's inequality, we can transform the criterion of $\theta$ minimisation (as in Eq. \eqref{eq:probs_introduced}) to a cross entropy loss. In Eq. \eqref{eq:probs_introduced} we essentially minimise $\E_{h,c} 1 - D_\theta (c \mid \Phi(h))$. If we remove constant terms and use Jensen's inequality, we arrive at a problem with equivalent solutions:

\begin{equation}
    \min_\theta - \E_{h,c} \log(D_\theta (c \mid \Phi(h)))
\end{equation}

If we estimate the expectation with $m$ samples, the criterion becomes the cross entropy loss. This suggests that we practically solve the same task that is solved when training state-of-the-art neural classifiers. However, the key differences are: we use the algorithm $\pi$ to create samples of a malicious class $\mal$ and instead of the classical mini-batch gradient descent \cite{minibatch_descent} we use Algorithm \ref{alg:detector_algorithm}.

\paragraph{On Complexity of Stochastic Detector}
The similarity to cross entropy imposes important implications. Since we practically use the same loss function but model a mixed strategy $\sigma$ instead of a single classifier $f$, the complexity of the detector $D_\theta$ needs to be much higher than the complexity of a classifier $f$. This also means, we shall expect training takes more time.

	\cleardoublepage
	\section{Experiments}\label{sec:experiments}
In previous sections, we proposed a theoretical approach to solve adversarial detection problems. Then we introduced an industrial problem which we formally modelled in accordance to the proposed theory. The outcome is an adversarial detector which when properly trained is able to detect unseen malicious activity. To support the claim we conduct experiments on real-world data (provided by Trend Micro Ltd.) and compare the proposed adversarial detector with an anomaly detector, both being attacked by the good queries attack and the gradient attack. The experiments evaluate: (1) capability of a detector to meet the false positive rate constraint, (2) capability of a detector to detect attacks (measured by the successful attacks rate), (3) performance of the proposed attack algorithms. Further, we analyse the results and identify that: (1) the dataset of benign data contains a few highly suspicious samples, (2) the adversarial detector is highly robust to primary goals with more than 10 low-scored URLs and (3) even more powerful attackers than those at train time are detected at relatively large rates.

\subsection{Dataset}
The problem of detecting malicious activity in requests to a URL reputation service was proposed by the company Trend Micro Ltd. as a real world problem that is an instance of adversarial machine learning. Thus, we use the company's data to evaluate the proposed algorithms in this work. The dataset we were, gratefully, given contains information that is, nonetheless, private and cannot be made public. Therefore, we do not put the dataset online. However, we are able to include general information and statistics to preview the properties of the dataset.

The dataset consists of genuine real-world activity recorded at such a URL reputation service. Users are uniquely identifiable, thus we are able to make up activity histories of each user. The users are located in the Czech Republic at the time of recording based on the IP address location.

First, we clean data by removing requests that are broken or their information is incomplete. These count: a queried URL is not a valid URL or it is missing. Then we remove requests with a URL that is not a genuine accessible URL: that are, for instance, URLs containing \texttt{.arpa} or \texttt{.in\_addr}. Then the activity is sorted to days and each queried URL is given a genuine reputation score returned by the reputation service. Then we collect these per-one-day per-user activity histories and remove those containing less then $10$ queries (i.e. 10 requests per user per day) for we assume an  activity this low is anomalous and including it would poisson the final data set.

The total number of samples after pre-processing is $54,970$ which split into a training set of $43$ thousand samples and a testing set of $11$ thousand samples (we use $80$-$20$ ratio). The reputation scores that are returned by the service are processed so that they correspond to a probability of a particular URL being benign. We are given only such values of the score so that the corresponding probability values are either $0.1$, $0.5$ or $0.9$. The dataset contains only few malicious URLs
($0.05\%$). The unrated URLs count $15.0\%$ and the benign ones $84.95\%$ of the URLs. Any future unknown (i.e. not included in the dataset) URL is considered unrated.

The distribution of URL use has very long tails. Fig. \ref{fig:url-frequency-histogram} shows roughly $50 \%$ of the URLs are used only once and $90 \%$ are used up to $10$ times. On the other hand, the dataset contains URLs that the service was queried with over $1,000$ times.

An average sample comprises an activity history of $700$ requests (per day). However, this distribution is fat-tailed. Most samples have around $1000$ requests, yet there are samples with over $10,000$ requests and, on the other hand, samples with $10$ requests. The histogram of the distribution is depicted in Fig. \ref{fig:requests-per-activity-histogram}.

In Fig. \ref{fig:request-time-histogram}, the request time distribution within a day is shown. Requests are sent mostly between $8$am and $10$pm with a peek between $6$pm and $9$pm. Also, there is a little drop around noon, suggesting this is a proper lunch time among recorded users. At night time from midnight until $7$am, there is a significant drop in the number of sent requests. Ideally, to disguise as a benign user, an attacker shall follow this distribution and adjusts its attack and obfuscation accordingly.

\begin{figure}[!htb]
    \centering

        \begin{subfigure}[b]{0.45\textwidth}
            \includegraphics[width=\textwidth]{request-time-histogram}
            \caption{}\label{fig:request-time-histogram}
        \end{subfigure}
        ~
        \begin{subfigure}[b]{0.45\textwidth}
            \includegraphics[width=\textwidth]{requests-per-activity-histogram}
            \caption{}\label{fig:requests-per-activity-histogram}
        \end{subfigure}

        \begin{subfigure}[b]{0.45\textwidth}
            \includegraphics[width=\textwidth]{score-histogram}
            \caption{}\label{fig:score-histogram}
        \end{subfigure}
        ~
        \begin{subfigure}[b]{0.45\textwidth}
            \includegraphics[width=\textwidth]{url-frequency-histogram}
            \caption{}\label{fig:url-frequency-histogram}
        \end{subfigure}

    \caption{The histograms show distributions of activity captured in March, 2019 among users of a URL reputation service, located in the Czech Republic. The dataset is provided by Trend Micro Ltd. Fig. \ref{fig:request-time-histogram} depicts request day-time distribution, Fig. \ref{fig:requests-per-activity-histogram} shows the amount of requests that is sent in one day activity of a user. Fig. \ref{fig:score-histogram} shows the distribution of a URL reputation score which is associated with a URL query. Finally, Fig. \ref{fig:url-frequency-histogram} shows the repetitive nature of such a reputation service.}
\end{figure}


\subsection{Setting}
To evaluate the training and attack algorithms we use the dataset provided by Trend Micro Ltd. In terms of attackers, we use the good queries attack algorithm (Sec. \ref{sec:good_queries_attack}) and the gradient attack algorithm (Sec. \ref{sec:gradient_attack}). To compare various approaches to detection, we use an anomaly detector based on $k$-NN (Sec. \ref{sec:knn_detector}) and an adversarial detector based on a neural network (Sec. \ref{sec:neural_detector}).

Since the detector's learning algorithm is proposed to learn against an attacking adversary, it is reasonable to consider the detector based on a neural network shall be trained against both attacker types. However, theoretically and empirically the good queries attack is not as advanced as the gradient attack. Thus we only perform experiments in which the neural net detector is trained vs. the gradient attack. On the other hand, we evaluate attack performance for both attack types.

\paragraph{False Positive Rate}
We train the neural net detector against the gradient attack algorithm and show results for false positive rate threshold values $\tau_0$ being $1\%$, $0.1\%$ and $0.01\%$ ($10^{-2}$, $10^{-3}$ and $10^{-4}$). FPR of $0.01\%$ on this size of a test set (to repeat, it counts $11$ thousand samples) corresponds to $1$ sample. This is already reaching limits of statistical evaluation and results with FPR set to values below $0.1\%$ would ideally require bigger dataset size to reach greater significance.

\paragraph{Attacker's Loss}
The attacker losses are set to correspond to motivations given in Sec. \ref{sec:attacker_loss}. That is, we set the cost of being detected $L_0$ to $100$ units and the cost of a request $L_u$ to $0.05$ units. The maximum attack cost is set to $99$ (motivation is given in Sec. \ref{sec:attacker_loss}). We set $L_u$ to 100 and the maximum attack cost to 99 to disallow the attacker to attack in extreme cases when its criterion closely reaches the no-attack threshold. This happens especially when the attacker's instance is in the area with very high confidence of malicious activity and it becomes rational no to carry out any obfuscation but to use only primary URLs $\pr{U}$. These cases are now labelled \NA.

\paragraph{Feature Map}
As already mentioned in Sec. \ref{sec:features}, we use four types of features. The first feature map is a frequency histogram of URL reputation score values contained in an activity history. The dataset comprises only three distinct values of a score, thus we use three bins with edges at $0$, $0.33$, $0.66$, $1$. The second feature type is URL score density which is a normed frequency histogram. The third feature is a square root of a total count of requests per activity history. The distribution of requests counts is fat-tailed, thus we reduce the influence of large values by taking a square root. We also tried  logarithm but square root seems to perform better. Lastly, we add a normalised frequency histogram of request times to relate to time distribution of requests. We use $24$ bins that each covers one hour. The full feature map counts $31$ features in total. Data are normalised so that each input has empirical mean equal to $0.0$ and variance equal to $1.0$ on training data.

\paragraph{Performance Measures}
To evaluate the detector's performance, we use two main measures: the first is a false positive rate (FPR), the second is successful attacks rate (SAR). The false positive rate is the rate of misclassification on benign data which shall, for the optimal detector, equal to the threshold $\tau_0$. Any deviation from the value $\tau_0$, negative or positive, is a failure because such a detector either does not meet the FPR criterion or is too benevolent, suggesting there is a tighter one with better detection rate.

The successful attack rate is the ratio of undetected attacks and the total number of attacks. SAR corresponds, in fact, to a false negative rate and is the main criterion of the detector's optimisation task (Prop. \ref{prop:detector_optimisation}). The obfuscation rate (OBR) is a subject of optimisation during the detector's learning (i.e. it is the false negatives rate). We define OBF as the ratio of undetected attacks and attacks performed (that is without \NA).
The lower the successful attack rate and the obfuscation rate, the better the detector is.
We also use other supporting measures: a \NA rate (NAR) and mean successful attack length (MAL). NAR gives a percentage of attacker instances that did not carry out any activity (and thus did not follow the goal). The mean successful attack length (MAL) gives an average number of additional URLs that were used in successful attacks (that is undetected attacks). To evaluate these measures, we use the as-if-deployed approach–-this means realisations of the probability $D_\theta(d \mid x)$ are drawn to make the final decision $d$. This suggests that the measured numbers are in fact realisations of random variables and thus those statistics are random variables as well.

\subsection{Detector Learning Procedure}

\paragraph{Lambda}
A key difference of the detector's learning algorithm (Alg. \ref{alg:detector_algorithm}) to standard classification learning schemes is the constraint on FPR which results in the variable $\lambda$ as a control variable. Lemma \ref{lem:langrangian_relaxation} proposes it is better to think of $\lambda$ as a priori class probability $p(\mal) = \frac{1}{1 + \lambda}$ which changes during the training procedure so that the constraint on FPR is met. We found that $p(\mal)$ very quickly converges to really low values (around $\sim0.95$) so that the FPR constraint is met.
To give little influence at least in the begging of learning to malicious data, we start training with $p(\mal) = 0.5$ (i.e. $\lambda = 1$). Once the FPR constraint is met and $FPR \leq \tau_0$, malicious data start again to have larger influence on gradient and $p(\mal)$ increases again.

\paragraph{Learning Rate}
We use gradient ascent to find $\lambda$ and gradient descent to find  parameters $\theta$. Since these are performed simultaneously but both correspond to different aspects of the problem, we found that their learning rates shall differ which goes along with suggestions in \cite{learning_rate}. We use a learning rate of magnitude $0.01$ in case of $\theta$ and a learning rate of $5.0$ in case of $\lambda$. This speeds up learning procedure especially in meeting the FPR constraint.

\paragraph{Batch Size}
In the process of gradient estimation, we generate $m$ samples for each class to get gradients conditioned on class. We call $m$ the batch size, although its meaning is different from the standard concept–-usually, a batch size refers to the number of samples drawn in total from a training set during a gradient descent step but we actually draw $m$ sample for each class (All introduced in Sec. \ref{sec:monte-carlo} and Sec. \ref{sec:learning_algorithm}). In terms of the value of the batch size, it turns out that the lower $m$ the greater the chance all of the drawn malicious instances generate \NA which essentially causes zero gradient attained on a malicious class. On the other hand, our set of primary goals (primary URLs) $T^\mal$ counts $360$ samples which after splitting creates a train set of 288 primary goals. Since we want to employ the mini-batch gradient descent \cite{minibatch_descent}, $m$ should be lower than the number of primary goals but reasonably large to suppress noise. To balance the two notions, we use $m=100$.

\paragraph{Attacker's Optimisation}
The obfuscation algorithm $\pi$ (Def. \ref{def:response_algorithm}) takes a primary URL set $\pr{U}$ and generates an obfuscated activity history $\adv{h}$ (or \NA) in $T$ steps. The gradient attack algorithm (Alg. \ref{alg:gradient_attack}) does so with a mixture of projected gradient descent (PGD) and fast gradient sign method (FGSM).
As introduced in Sec. \ref{sec:gradient_attack_algorithm}, we set the initial time distribution of activity $k^B$ to be uniform and the initial number of obfuscation URLs $k^A$ randomly to any of $\{0, 1, \dots, 2000 \}$. Naturally, the optimal number of steps $T$ balances the quality of a solution and time needed to find it. We use $T = 400$ which turns out to be a reasonable balance.

\subsection{Results}
In this section, we show that the problem of detecting malicious behaviour is better solved with an adversarial detector which outperforms an anomaly detector. First, we show performance of the detectors in various settings and analyse their exploitability against two attack types. Secondly, we analyse the attacks performed against the best detector and show what kinds of attacks are are very likely to be detected.

\subsubsection{Optimal Detector}\label{sec:optimal_detector}
To address the problem of detecting malicious users of a URL reputation service, we use two models of a detector: an anomaly detector based on $k$-NN and an adversarial stochastic detector based on a neural net. We show that the neural net trained against a model of an attacker outperforms the anomaly detector. We use three FPR thresholds, $\tau_0 \in \{ 1\%, 0.1\%, 0.01\% \}$, and perform attacks with the good queries attack and the gradient attack.

\paragraph{False Positive Rate Threshold}
A key requirement of a detector in network security is that the false positive rate (FPR) is below a threshold $\tau_0$ (as argued in Sec. \ref{sec:property_one}). To validate our detectors are able to meet this constraint, we fit them on train data and measure FPR on test data. The adversarial detector is fitted against the gradient attack. The FPR results are shown in Tab. \ref{tab:false_positives_rate}. All training sessions successfully converge below a desired threshold value. However, the anomaly detector does not meet the FPR constraint on a test set in case of $\tau_0 = 1\%$ and $\tau_0 = 0.1\%$, whereas the adversarial succeeds in all settings.

Note that the training FPR is usually below the desired threshold. Take for instance the anomaly detector that with $\tau_0 = 1\%$ gives a training FPR $ 0.85 \%$.
This is caused, we argue, by the distribution of benign data. The distribution contains a relatively large number of outliers that sparsely located far from the distribution. This causes a detector with fixed limited complexity (i.e. k in $k$-NN) is not capable of reaching the exact value of $tau_0$.

\begin{table}[h]
\centering

    \begin{tabular}{|l||c|c|c|c|c|c|}
    \hline
    FPR Threshold $\tau_0$        & \multicolumn{2}{c|}{$1\%$}              & \multicolumn{2}{c|}{$0.1\%$}            & \multicolumn{2}{c|}{$0.01\%$}           \\ \hline
                              & \multicolumn{1}{l|}{$k$-NN} & AdvDet & \multicolumn{1}{l|}{$k$-NN} & AdvDet & \multicolumn{1}{l|}{$k$-NN} & AdvDet \\ \hline\hline
    FPR on Train Data & 0.58\% & 0.96\% & 0.09\% & 0.07 \% & 0.01\% & 0.01\% \\ \hline
    FPR on Test Data & 1.03\% & 0.96\% & 0.11 \% & 0.08\% & 0.01 \% & 0.01\% \\ \hline
    \end{tabular}

    \caption{False Positive Rates on Test Data}
    \label{tab:false_positives_rate}

\end{table}


\begin{table}[h]
\centering

    \begin{tabular}{|l||c|c|c|c|c|c|}
    \hline
    FPR Thresh. $\tau_0$        & \multicolumn{2}{c|}{$1\%$}              & \multicolumn{2}{c|}{$0.1\%$}            & \multicolumn{2}{c|}{$0.01\%$}           \\ \hline
                              & \multicolumn{1}{l|}{$k$-NN} & AdvDet & \multicolumn{1}{l|}{$k$-NN} & AdvDet & \multicolumn{1}{l|}{$k$-NN} & AdvDet \\ \hline\hline

    NAR & 52\% & 57\% & 54\% & 32\% & 45\% & 1 \% \\ \hline
    OBR & 25\% & 3\% & 85\% & 39\% & 95\% & 49\% \\ \hline
    SAR & 12\% & 1 \% & 39\% & 27\% & 52\% & 48\% \\ \hline

    \end{tabular}

    \caption{Gradient Attack Results}
    \label{tab:gradient_attack_results}

\end{table}

\begin{table}[h]
\centering

    \begin{tabular}{|l||c|c|c|c|c|c|}
    \hline
    FPR Thresh. $\tau_0$        & \multicolumn{2}{c|}{$1\%$}              & \multicolumn{2}{c|}{$0.1\%$}            & \multicolumn{2}{c|}{$0.01\%$}           \\ \hline
                              & \multicolumn{1}{l|}{$k$-NN} & AdvDet & \multicolumn{1}{l|}{$k$-NN} & AdvDet & \multicolumn{1}{l|}{$k$-NN} & AdvDet \\ \hline\hline
    NAR & 88\% & 94\% & 54\% & 4\% & 0\% & 0\% \\ \hline
    OBR & 44\% & 50\% & 78\% & 13\% & 98\% & 32\% \\ \hline
    SAR & 5\% & 3\% & 36\% & 12\% & 98\% & 32\% \\ \hline
    \end{tabular}

    \caption{Good Queries Attack Results}
    \label{tab:good_queries_attack_results}

\end{table}

\paragraph{Exploitability by Gradient Attack}
We measure exploitability with a successful attack rate (SAR). The adversarial detector gives better performance than the anomaly detector in terms of SAR. The results of the gradient attack algorithm are shown in Tab. \ref{tab:gradient_attack_results} and the trend is shown in Fig. \ref{fig:gradient-attack-fpr-sar}. With the FPR threshold at $1\%$, successful attack rate (SAR) is at 12\% for the anomaly detector ($k$-NN) and 4\% for the adversarial detector.
But with lower thresholds, the difference narrows. At $0.1 \%$, the anomaly detector at $39 \%$ and the adversarial detector allows SAR at $32 \%$. With FPR threshold $\tau_0 = 0.01 \%$, the adversarial detector (SAR $48 \%$) outperforms the anomaly detector (SAR $52 \%$). But both detectors reach almost 50\% exploitability. To sum it up, the adversarial detector outperforms the anomaly detector, especially at 1\% threshold.

We assume that the narrowing performance margin is also given by a low number of benign samples for this low FPR thresholds. As mentioned, for the 0.01\% threshold, the detector achieves the desired constraint by misclassifying at most only 5 benign samples among the outliers in the train set. We conjecture that with a greater dataset, the adversarial detector achieves lower SAR values even with $\tau_0 = 0.01\%$.

\begin{figure}[h]
    \centering

    \begin{subfigure}[b]{0.45\textwidth}
        \includegraphics[width=\textwidth]{gradient_attack_success}
        \caption{Successful Attack Rate (SAR) of Gradient Attack}
        \label{fig:gradient-attack-fpr-sar}
    \end{subfigure}
    ~
    \begin{subfigure}[b]{0.45\textwidth}
        \includegraphics[width=\textwidth]{good_queries_success}
        \caption{Successful Attack Rate (SAR) of Good Query Attack}
        \label{fig:good_queries-attack-fpr-sar}
    \end{subfigure}


    \caption{Successful attack rate (SAR) as a function of the false positive rate (FPR). Note that, as the FPR threshold is increased, both detectors become more robust. At all FPR levels and with both attack types, the adversarial detector outperforms the anomaly detector.}

\end{figure}

\paragraph{Exploitability by Good Queries Attack}
The good queries attacks algorithm adds legitimate obfuscating requests to the final activity history as long as the cost of an attack decreases. In comparison to the gradient attack, it is weaker but more realistic to capture behaviour of a malicious user that rather intuitively obfuscates its primary goal. In Tab. \ref{tab:good_queries_attack_results} we show the results of the good queries attack against anomaly and adversarial detectors at various levels of FPR. The overall trend is depicted in Fig. \ref{fig:good_queries-attack-fpr-sar}. It is clear that, same as with the gradient attack, the adversarial detector allows fewer attacks in at all FPR levels.

The difference is best seen at $\tau_0 = 0.1\%$ where the anomaly detector achieves a successful attack rate (SAR) of 36\%, whereas the adversarial detector achieves 12\%. Note that, with $\tau_0 = 0.01\%$ and the anomaly detector, the good query attack (SAR 98\%) performs surprisingly better than the gradient attack (SAR 52\%). This is caused by the large NAR of 45\% in the gradient attack while it is 0\% with the good query attack -- this means a large portion of attacks was not carried out due to, probably, too few iterations. However, if we consider the obfuscation rate (OBR) which is the percentage of successful undetected obfuscations out of actually performed attacks, both attacks are nearly similar with this detector.

To conclude, both detectors are robust against the good queries attack but the adversarial detector allows lower attack success rates at all FPR levels.


\paragraph{Suspicious Outliers in Dataset}
The set of benign activity $T^\ben$ comprises data of real users of the company Trend Micro Ltd. During the training process of our detector, some benign samples tended to be classified as malicious with relatively high confidence (over $90\%$). A closer inspection revealed these samples truly contain requests with URLs that are suspicious. In fact, a single user was repeatedly labelled malicious in two of its sample (i.e. two independent days of activity). For instance, this user queried the service with a URL of a domain which, when visited, redirects to google.com if no URI path is given. But once a specific and long URI path is appended to the domain, it instead redirects several times to various other domains and gives an empty site in the end. Of course, this is far from identifying this particular user is a true malicious actor - it very well may have been an infected computer - but it shows that the detector correctly labels samples that contain suspicious activity and considers them outliers.

\subsubsection{Attack Analysis}\label{sec:attack_analysis}

\begin{figure}[p]
    \centering

    \includegraphics[width=0.95\textwidth]{attacks-trend}
    \caption{Primary Goals – \NA Dependency. We found that there is an emerging pattern in the obfuscation ability against the adversarial detector. Some primary goals tend to be too costly to be obfuscated so the algorithm turns them to \NA. This figure shows the pattern: we draw primary goals (primary URL sets) that are modified to \NA in red and primary goals that are turned into an activity history in blue. The axes correspond to parameters of the primary goals we generated: the x-axis shows a number of URLs with a malicious reputation score in a primary goal $\pr{U}$; the y-axis shows the number of unrated URLs. The figure depicts individual primary goals as dots and an estimated density distributions with contours. Clearly, primary goals with more $10$ malicious URLs tend to become \NA whereas primary goals with fewer than $10$ malicious URLs tend to be conversed to an obfuscated activity history and a corresponding attack is carried out. All data are from a test set of primary goals and the results are taken from an attack against the adversarial detector with $FPR = 0.1\%$.}\label{fig:attack-trends}

\end{figure}

\paragraph{Primary Goal - No-Attack Dependency}

\begin{figure}[p]
    \centering

        \begin{subfigure}[b]{0.45\textwidth}
            \includegraphics[width=\textwidth]{attack-400}
            \caption{Attack with $T = 400$ and $L_u = 0.05$}\label{fig:attack-400}
        \end{subfigure}
        ~
        \begin{subfigure}[b]{0.45\textwidth}
            \includegraphics[width=\textwidth]{attack-800}
            \caption{Attack with $T = 800$ and $L_u = 0.05$}\label{fig:attack-800}
        \end{subfigure}

        \begin{subfigure}[b]{0.45\textwidth}
            \includegraphics[width=\textwidth]{attack-800-smaller-cost}
            \caption{Attack with $T = 800$ and $L_u = 0.005$}\label{fig:attack-800-smaller-cost}
        \end{subfigure}
        ~
        \begin{subfigure}[b]{0.45\textwidth}
            \includegraphics[width=\textwidth]{attack-1600}
            \caption{Attack with $T = 1600$ and $L_u = 0.0005$}\label{fig:attack-1600}
        \end{subfigure}

    \caption{The figures plot single activity histories as points in a feature space. All images are PCA-transformations of the feature space with identical principal components. Black points are \NA, red points are obfuscation activity of individual attacker instances, blue points are benign activity histories. The contours correspond to the class posteriori probability modelled with the detector. All pictures show the results of the adversarial detector. Upper left, the results of the gradient attack with $400$ iterations – NAR is $57.33\%$. Upper right, attack with $800$ iterations - NAR is $29.33\%$. Bottom left, attack with $800$ iterations but less expensive per-request cost, $L_u = 0.005$. Bottom right, attack with $1600$ iterations but extremely inexpensive per-request cost, $L_u = 0.0005$. Note that attacks are located in different areas as we change attacker's costs. In the extreme case (bottom right), nearly all attacks are located beyond the detector's contours.}\label{fig:attack-iterations}
\end{figure}

As argued, the problem of malicious activity detection is difficult in that only benign data are available at the time of training. In this work, we proposed a model of an attacker and, consequently, created a feasible set of primary goals. The training set of malicious data $T^\mal$ contains primary URLs sets $\pr{U_i}$ that we crafted purposely to represent various attackers. We generated $\pr{U_i}$ to comprise URLs with a malicious reputation score and URLs with a yet unrated reputation score. (All introduced in Sec. \ref{sec:training_set})

We trained an adversarial detector and then performed attacks with the gradient attack algorithm. We found that there is a pattern in what primary goals tend to get obfuscated and what are turned into \NA. The relation is depicted in Fig. \ref{fig:attack-trends} where we plot each test set primary goal as a single point parametrised by the contents of the primary goal (i.e. the primary URL set $\pr{U_i}$). Primary goals that are turned to \NA are coloured in red and primary goals that are obfuscated and reassemble an attack are coloured blue. To outline the pattern, we estimate probability density for each group: \NA and attacks. The figure shows that primary URLs sets containing more than roughly $10$ truly malicious URLs are more likely to become \NA. Whereas primary URLs sets with less than $10$ truly malicious URLs are prone to become an obfuscated activity history.

This means that the optimal detector is more exploitable by attacker instances that have fewer truly malicious URLs in their primary goals whereas an attacker instance with a lot of truly malicious URLs tends to be detected. This implies that, after deploying this detector, an attacker which has full knowledge of the setting deals with a fact that employing more than ten malicious URLs a day leads to too costly obfuscation and it is rational not to attack.

\paragraph{\NA Rates}
The attacks performed in our experiments tend to have large \NA rates (usually NAR is between $50\%$ and $80\%$). A \NA occurs if, given a primary goal, the attack algorithm fails to create an obfuscation activity history because carrying out the attack is too costly. However, it seems this happens too often due to imperfections of the attack algorithm: it for example gets stuck in a local optimum or convergence takes too many iterations. We argue relatively high NAR is mainly caused by the number of iterations of the attack algorithm which, despite already being high ($400$), is sometimes insufficient for finding a less costly activity history. This can be seen when we attack a fitted detector with a gradient attack that runs in $800$ iterations. By doubling the number of iterations, NAR drops from $57.33\%$ to $29.33\%$. Interestingly enough, these attacks, nonetheless, maintain comparable OBR and SCR, i.e. detector's exploitability remains unchanged even though the attacker uses more iterations to craft the attack. Fig. \ref{fig:attack-400} shows attacks of a 400-iterations attacker and Fig. \ref{fig:attack-800} shows a 800-iterations attacker. The figures depict a PCA-transformed feature space with test data of both benign and malicious classes. The red points are final malicious attacks, while the black points are primary goals turned to \NA. The contours show a class posteriori probability modelled by the detector, but projected to a hyperplane attained by PCA. Note that the original problem has over 30 dimensions, thus this view skews distances and causes some relations are misleading. However, the main point can be illustrated: when attacking with 800 iterations the attacker converts more primary goals to an activity history than with 400 iterations.

\paragraph{Attack Cost Analysis}

\begin{table}[t]
\centering

    \begin{tabular}{|l|c|c||c|c|c|c|}
    \hline
                          & $L_u$     & Iterations & NAR      & SAR      & OBR      & MAL      \\ \hline\hline
    Train Time Attack     & $0.05 \%$ & $400$   & $57\%$ & $1 \%$ & $3 \%$ & $20$ \\ \hline
    More Iterations       & $0.05 \%$ & $800$   & $29 \%$ & $1 \%$ & $1 \%$ & $19$ \\ \hline
    Cheaper Requests      & $0.005 \%$ & $800$   & $5\%$ & $29\%$ & $30 \%$ & $146$ \\ \hline
    The Cheapest Requests & $0.0005 \%$ & $1,600$   & $0 \%$ & $94 \%$ & $94 \%$ & $2003$ \\ \hline
    \end{tabular}

\caption{Attack Cost Analysis}
\label{tab:attack-cost-analysis}

\end{table}

The attacker's loss has two constants: a cost for being detected $L_0$ and a cost for sending one request $L_u$. As presented in Sec \ref{sec:attacker_loss}, we use $L_0 = 100$ and $L_u = 0.05$. To repeat, per-request cost $L_u = 0.05$ can be interpreted as follows: an activity history that is labelled as $0\%$ malicious costs exactly $L_0$ when it contains $2,000$ additional obfuscation URLs. Or to put it differently, an obfuscation activity may contain at most 2, 000 additional requests. However, as we pointed out in the dataset analysis above, the median of the number of requests per activity history is actually $\sim 1,000$ and the distribution contains well-represented activity histories even with $\sim 5,000$ requests.
Thus by this choice, we limit the attacker to create activity histories with fewer than 2000 requests which, however, we argue is a reasonable amount for an attacker.

To check whether attacks with lower per-request cost $L_u$ are able to circumvent the detector, we perform attacks with cheaper costs. In addition: during attack generation, we use $400$ iterations in the attack's gradient algorithm which in each iteration changes the number of requests by little (usually the change is $2$ or $3$ requests). Thus, a smaller $L_u$, which implies higher maximal number of requests, necessarily requires more iteration steps – which is costly and causes significant increase in training time.

From the reasons above, we attack the adversarial detector with the gradient attack with: 400 iterations and $L_u = 0.05$ (train time attack), 800 iterations and $L_u = 0.05$ (more iterations), 800 iterations and $L_u = 0.005$ (cheaper requests) and 1,600 iterations and $L_u = 0.0005$ (the cheapest requests). The attacks' results can be seen in Tab. \ref{tab:attack-cost-analysis}. The attacks are depicted in Fig. \ref{fig:attack-iterations}.

As discussed above, doubling the iterations number from 400 to 800 maintains the adversarial detector's exploitability. However, if we lower the per-request cost to $L_u = 0.005$ which changes the maximum number of requests to 20,000, the detector's exploitability suffers. The successful attack rate (SAR) increases to $30\%$ from $1 \%$. Following the lower cost of requests, the mean attack length increases as well from 20 to 146. OBR rises to $30.99\%$ with NAR at $5.33\%$. This also shows that attacks that were carried out previously at higher costs are now turned into an activity history with greater chance of a successful obfuscation. However, the values are not critical and are comparable to the values attained on an anomaly detector with a train time attack ($L_u = 0.05$ and 400 iterations.).

The resulting attacks of this setting can also be seen in Fig. \ref{fig:attack-800-smaller-cost}. Note that points representing the attacks moved towards the detector's contours but they remained in the malicious-labeled area. We assuem the shift in attack placements reflects the lower per-request cost as the attacker is able to mix in more obfuscation URLs and move closer to legitimate benign samples with occasional low-score URL appearances.

Finally, an attack with $L_u = 0.0005$ (equivalent to $200,000$ max-requests number) and the number of iterations $1,600$ increases OBR rapidly to $94.67 \%$ while entirely erasing NAR ($ 0.0\%$). Accordingly, MAL increases to $2,000$. The detector is largely exploitable by this attack. Same is seen in Fig. \ref{fig:attack-1600} which depicts attacks with the cheapest per-requests cost. Most of the attacks are now moved pass the detector's contours to the area of the feature space with high benign posteriori probability. Note that compared to the previous attacks, the cheapest cost attack instances are placed in a different area - higher along the y-axis. This corresponds to the fact that obfuscation may occur with more requests creating an activity history with mean attack length around 2,000. These activity histories were not generated during training as the train time attacker was limited to at most 2,000 additional requests. Therefore, we may expect that this is a blind spot of the detector because it was not trained to detect such attacks.

It is important point out that the last attack (the cheapest cost) is off scale compared to the attack used at train time. In addition, we conjecture that a detector becomes robust even to this attack if it is trained against it using the detector's learning algorithm (that is following the same procedure but with much lower attacker costs and more iterations during the attack). This, however, will increase computational requirements and training time.

\paragraph{Anomaly Detector - Adversarial Detector Comparison}
As argued, an anomaly detector does not incorporate a model of an attacker which means it defends attacks "from all directions". Whereas, an adversarial detector takes advantage from the attacker model and defends only "directions which are susceptible to lure attacks". This is better seen in Fig. \ref{fig:contours-comparison} which depicts a view of the feature space transformed with PCA. The figure shows contours of detectors' posterior class probability $D_\theta(\mal | x)$. In case of the adversarial detector, we see that the detector's contours are shaped to reflect areas in which an attack is likely and entirely omit areas in which attacks are not found. Whereas, in case of the anomaly detector, contours only reflect the benign data distribution. This is the main advantage given to an adversarial detector - it models its posterior probability so that it reflects possible attacks.

\begin{figure}[p]
    \centering

        \begin{subfigure}[b]{0.45\textwidth}
            \includegraphics[width=\textwidth]{adv-grad-attack}
            \caption{Adversarial Detector}\label{fig:adv_grad_attack}
        \end{subfigure}
        ~
        \begin{subfigure}[b]{0.43\textwidth}
            \includegraphics[width=\textwidth]{knn-grad-attack}
            \caption{Anomaly Detector}\label{fig:knn_adv_attack}
        \end{subfigure}

    \caption{The figures plot single activity histories as points in a feature space. All images are PCA-transformations of the feature space with identical principal components. Black points are \NA, red points are obfuscation activity of individual attacker instances, blue points are benign activity histories. The contours correspond to the class posteriori probability modelled with the detector. The pictures compare the shape of detector's posterior class probability that is depicted wtih contours. On left, the adversarial detector shapes its contours to reflect possible attacks, whereas, on right, the anomaly detector omits the attacker's model and thus adjusts its contours to face all possible anomalies. This gives the adversarial detector advantage in that it reflect the attck distribution.}\label{fig:contours-comparison}
\end{figure}

	%%\addcontentsline{toc}{section}{Conclusions}
	\cleardoublepage
	\section*{Conclusions}
This work examined adversarial machine learning in network security. We focused on a problem of detecting malicious activity while, ideally, not affecting benign users. We started with the assumptions that a detection false positive rate is constrained by a threshold; and that malicious activity cannot be recorded faithfully and, in addition, changes in response to the parameters of a detector. (Sec. \ref{sec:specifics_of_aml})

To model the setting, we modified the empirical risk minimisation framework to correspond to the Neyman-Pearson task (Sec. \ref{sec:property_one}). Using the notion of statistical learning, we defined a game in which a detector identifies the best parameters of its stochastic detection classifier and an attacker searches for the optimal obfuscation method of its primary goals (Sec. \ref{sec:stackelberg_game}). Such a model of an attacker builds on the assumption that malicious actors aim to act rationally by minimising their risk (Sec. \ref{sec:attacker-def}).
In terms of the detector, we argued that it necessarily must be a stochastic detector which means that instead of finding a classifier we solve a task of modelling the posterior class probability and draw the final label from it. (Sec. \ref{sec:stochasticity_importance})

Then, assuming the players play a strong Stackelberg equilibrium, we arrived at a bilevel optimisation problem whose solution is a robust detector that minimises exploitability by an attacker and keeps its false positive rate below a given threshold (Sec. \ref{sec:stochastic_detector}). We then proposed a detector's learning algorithm that approximates the solution of the optimisation problem and outputs a detector which despite being trained solely on legitimate benign data is able to detect unseen malicious activity. The detector's learning algorithm follows a scheme of adjusting parameters according to the attacker's best responses (Sec. \ref{sec:detector_learning_algorithm}).

In Sec. \ref{sec:anomaly_detection}, we proposed that the task of detecting malicious activity can be also solved with an anomaly detector. This approach comes with a critical disadvantage – a model of an attacker is omitted entirely and the anomaly detector can be trained only on benign data.

As a running example, we used attacks to a URL reputation system and proposed a formal model of those attacks (Sec. \ref{sec:formal_definition}). We proposed a good query attack (Sec. \ref{sec:good_queries_attack}) and a gradient attack (Sec. \ref{sec:gradient_attack}). The gradient attack, given a primary goal (a set of target URLs), identifies a local-optimum activity that obfuscates this primary goal. The attack is novel in that it is able to create obfuscating activity even in domain that is highly discrete.
This is achieved because we conveniently parametrised attacks and used projected gradient descent and a fast gradient sign method to find the local-optimum set of requests (Sec. \ref{sec:attack_parametrisation}).

Then we adjusted the proposed detector's learning algorithm to the domain of reputation score requests. We proposed a neural network architecture with five layers that models a posterior class probability (Sec. \ref{sec:neural_detector}).

Using real-word data provided by Trend Micro Ltd., we show that an adversarial detector outperforms an anomaly detector in all false positive ratio scenarios (1\%, 0.1\% and 0.01\%). In terms of meeting the false positive rate constraint, an adversarial detector meets it in all three scenarios, whereas the anomaly detector meets the constraint only on training data. In terms of exploitability (which we measure in successful attack rates), an adversarial detector detects more attacks than an anomaly detector. (Sec. \ref{sec:optimal_detector})

Concerning real-time benign data, we found that it contains outliers which our detector confidently labels malicious. On closer manual inspection, those outliers seem to carry out suspicious activity and are, thus, correctly labeled malicious (Sec. \ref{sec:attack_analysis}).

In future work, a detector's learning algorithm convergence proof may be delivered. Also, our feature map omits sequential nature of activity histories based on sent request-a more complex feature map that takes raw request sequences may be researched.

In conclusion, this work proposed a theoretical background and a practical algorithm to a problem of detecting malicious activity in network security. We like to think of it as a proof-of-concept for real-world adversarial detection problems.

\end{headerline}


	\cleardoublepage
	\bibliography{master_thesis_references}{}

	\cleardoublepage
	\listoffigures

	\cleardoublepage
	\listoftables



%%%%%%%%%%%%%%%%%%%% Appendices




\end{document}
